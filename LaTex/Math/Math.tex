\documentclass[12pt, a4paper, oneside]{ctexbook}
\usepackage{amsmath, amsthm, amssymb, bm, graphicx, hyperref, mathrsfs}



\setlength{\parindent}{0pt}  
\title{{\Huge{\textbf{XCPC 数学基础}}}}
\author{ljh}
\date{\today}
\linespread{1.5}
\newtheorem{theorem}{定理}[section]
\newtheorem{definition}[theorem]{定义}
\newtheorem{lemma}[theorem]{引理}
\newtheorem{corollary}[theorem]{推论}
\newtheorem{example}[theorem]{例}
\newtheorem{proposition}[theorem]{结论}
\newcommand{\newp}[1]{\vspace{#1\baselineskip}\noindent}  

\begin{document}

\maketitle

\pagenumbering{roman}
\setcounter{page}{1}

\newpage
\pagenumbering{Roman}
\setcounter{page}{1}
\tableofcontents
\newpage
\setcounter{page}{1}
\pagenumbering{arabic}

\chapter{组合数学}
\section{基本排列组合公式}
1. 线性排列:n个数的r排列 $P(n,r)=\frac{n!}{(n-r)!}$\\
2. 圆排列:n个数的r排列 $\frac{P(n,r)}{r}$\\
3. 项链数:$n$ 个不同的珠子串成一串项链,则得到不同的项链数为
$$
p=\begin{cases}
  1,& \left( n<=2\right) \\
  \frac{(n-1)!}{2},& \left( otherwise\right)
  \end{cases}
$$
4. 多重集合的排列: 有$k$ 种元素,每种$n_1,n_2,\dots n_k$个,的排列公式为
$$
\frac{n!}{\prod_{i=1}^{k}\left(n_i!\right)} 
$$
或记为
$$
\binom{n}{n1,n2,\cdots,n_k} 
$$
5. 组合: 
$$
\binom{n}{r}=\frac{P(n,r)}{r!} 
$$

6. 多重集的组合
设$S$是有$k$ 种元素的集合,每种元素无限个$(\geq r)$,则其$r$ 组合的个数为:
$$
\binom{r+k-1}{r} 
$$
或者说有
\begin{proposition}
$x_1+x_2+\dots+x_k=r(x_i\geq 0)$的整数解有
$$
\binom{r+k-1}{r} 
$$
种.
\end{proposition}

\newp{3}

\section{重要组合恒等式}
1. Pascal公式
$$
\binom{n}{k}=\binom{n-1}{k}+\binom{n-1}{k-1} 
$$

2.
$$
\sum_{i=0}^{n} \binom{n}{i}=2^n 
$$

3. 
$$
 m\cdot \binom{n}{m}=n\cdot \binom{n-1}{m-1}
$$

4.
$$ 
\sum_{k=1}^{n} k\cdot\binom{n}{k}=n\cdot \sum_{k=1}^{n}\binom{n-1}{k-1}=n\cdot2^{n-1} 
$$

5. 朱世杰恒等式
$$
\binom{m+n+1}{n+1}=\sum_{i=0}^{m} \binom{n+i}{n}
$$

6.范德蒙德恒等式
$$
\binom{a+b}{n}=\sum_{i=0}^{k}\binom{a}{i}\binom{b}{n-i}
$$

特别地:
$$
\binom{2n}{n}=\sum_{i=0}^{n}\binom{n}{i}\binom{n}{n-i}
$$

\begin{proposition}
$m$ 个$a$,和最多$n$ 个b的排列数等于
$$
\binom{m+n+1}{m+1} 
$$
\end{proposition}

\begin{proposition}
  最多 $m$ 个$a$,和最多$n$ 个b的排列数等于
  $$
  \binom{n+m+2}{m+1}-1
  $$
\end{proposition}

7.
$$
\sum_{1\le k \le n} k\binom{n}{k} =n2^{n-1} \qquad (n\ge 1)
$$
8.
利用导数可以得到
$$
\sum_{1\le k \le n} k^2 \binom{n}{k} =n(n+1)2^{n-2}  \qquad (n\ge 1)
$$

\newp{3}

\section{二项式系数}

\begin{proposition}
在杨辉三角中规定只能向下或者右下移动,从(0,0)到(n,k)的路径数为$\binom{n}{k}$
\end{proposition}

\begin{theorem}
二项式定理
$$
(x+y)^n=\sum_{k=0}^{n}\binom{n}{k} x^{n-k}y^k
$$
\end{theorem}

\begin{theorem}
Sperner定理:\\
设$S$是$n$ 元素集合.那么$S$上的一个反链至多包含$\binom{n}{\lfloor \frac{n}{2} \rfloor} $个集合.\\
其中,这里的反链指的是以集合包含为偏序关系的反链,即$S$ 的一个子集的集合,任何两个集合没有关系.
\end{theorem}

\begin{proposition}
多项式系数的帕斯卡公式
$$
\binom{n}{n_1,n_2,\cdots,n_t}=\binom{n-1}{n_1-1,n_2,\cdots,n_t}+\binom{n-1}{n_1,n_2-1,\cdots,n_t}+\cdots+\binom{n-1}{n_1,n_2,\cdots,n_t-1}
$$
\end{proposition}

\begin{theorem}
多项式定理
$$
(x_1+s_2+\cdots+x_t)^n=\sum \binom{n}{n_1,n_2,\cdots,n_t}x_1^{n_1}x_x^{n_2}\cdots x_t^{n_t}
$$
\end{theorem}

\begin{theorem}
牛顿多项式定理
  $$
  (1+z)^a=\sum_{k=0}^{\infty} \binom{a}{n} z^k \qquad  (a\in R,|z|<1)
  $$
\end{theorem}

\begin{theorem}
Dilworth定理\\
设$(X,\leq)$是有限偏序集合,而$m$是反链的最大大小,则$X$可以被划分为$m$个链,但不能被划分成小于$m$个链.\\
设$(X,\leq)$是有限偏序集合,而$r$是链的最大大小,则$X$可以被划分为$r$个反链,但不能被划分成小于$r$个反链.
\end{theorem}
\newp{3}
\section{抽屉原理}
简单形式
\begin{proposition}
如果要把$n+1$个物体放进$n$个盒子,那么至少有一个盒子有至少$2$个物体
\end{proposition}
加强形式

\begin{proposition}
设$q_1,q_2,\dots q_n$ 是正整数.如果将$q_1+q_2+\dots +q_n-n+1$个物体放进$n$ 个盒子.那么要么要么第一个盒子含有$q_1$个物体,$\dots$ ,要么第$n$ 个物体含有$q_n$个物体.
\end{proposition}

\begin{theorem}
Ramsey定理\\
在$6$个人(或者更多),要么有$3$个人互相认识,要么有$3$个人互相都不认识.\\
或者说\\
对于$K_n(n \geq 6)$我们给他的所有边染红色或蓝色,总存在一个红$K_3$ 或蓝$K_3$,记为$K_6\rightarrow K_3,K_3$
\end{theorem}

推广 
\begin{theorem}
若$m,n \geq2 $,存在正整数$p$,使得$K_p\rightarrow K_m,K_n $
事实上,注意到若$p$成立,则对于$q\geq p$都成立,取一个子图即可.
我们记Ramsey数$r(m,n)$为使之成立的最小的数.Ramsey定理保证这样的数一定存在.
注意到$$r(m,n)=r(n,m)$$
以及$$r(2,m)=m $$ 当$m\geq2$时,$r(2,m)$称为平凡的Ramsey数(交换同理).
\end{theorem}

性质\\
1. $$r(m,n)\leq r(m-1,n)+r(m,n-1)(m,n \geq 3) $$
2. $$r(m,n)\leq \binom{m+n-2}{n-1} $$(数学归纳法证明)
\newp{3}

\section{容斥原理}
\subsection{容斥原理}
\begin{theorem}
容斥原理
$$
|\overline{A_1}\cap \overline{A_2} \cap \cdots \cap \overline{A_n}|=|S|-\sum|A_i|+\sum|A_i\cap A_j|+\cdots+(-1)^n|A_1\cap A_2\cap \cdots \cap A_n|
$$
\end{theorem}
可根据贡献法证明.\\
应用:不定方程整数解个数问题
\begin{example}
  求下列方程整数解个数
  $$
  x_1+x_2+x_3+x_4=18
  $$
  满足
  $$
  1\leq x_1 \leq 5 ,\quad -2\leq x_2\leq4,\quad0\leq x_3\leq 5,\quad 3\leq x_4\leq 9
  $$
  解:\\
  等价于
  $$
  a_1+a_2+a_3+a_4=16
  $$
  满足
  $$
  0\leq a_1 \leq 4 ,\quad 0\leq a_2\leq6,\quad0\leq a_3\leq 5,\quad 0\leq a_4\leq 6
  $$
  不加范围的解的个数为
  $$
  |S|=\binom{16+4-1}{16}=969
  $$
  其中设$A_1$为$a_1$大于$4$的解的集合$A_2$为$a_2$大于6的解的集合$\cdots$
  $$
  \begin{aligned}
  |A_1|=&\binom{11+4-1}{11}=364\\
  |A_2|=&\binom{9+4-1}{9}=220\\
  |A_3|=&\binom{13}{10}=286\\
  |A_4|=&\binom{12}{9}=220\\
  \end{aligned}
  $$
  同理算交集.然后根据容斥原理可得出答案为$55$
\end{example}

\begin{proposition}
  错位排列
  $$
  D_n=n!\left(1-\frac{1}{1!}+\frac{1}{2!}-\frac{1}{3!}+\cdots+(-1)^n\frac{1}{n!}\right)
  $$
\end{proposition}
有性质\\
1.$$\frac{D_n}{n!}\approx e^{-1}$$
2.
$D_1=0,D_2=1$
$$
D_n=(n-1)(D_{n-1}+D_{n-2})
$$
3.
$$
\begin{aligned}
&D_n=(n-1)(D_{n-1}+D_{n-2})\\
\iff&D_n-nD_{n-1}=-(D_{n-1}-(n-1)D_{n-2})\\
\iff&D_n=nD_{n-1}+(-1)^n
\end{aligned}
$$

\begin{proposition}
$$ 
 Q_n=n!-\binom{n-1}{1}(n-1)!+\binom{n-1}{2}(n-2)!-\binom{n-1}{3}(n-3)!+\cdots+(-1)^{n-1}\binom{n-1}{n-1}1!
$$
\end{proposition}
为不出现$i(i+1)$的排列数
并且有
$$
Q_n=D_n+D_{n-1}
$$
\subsection{莫比乌斯反演}
容斥原理是莫比乌斯反演在有限偏序集上的一个实例.\\
偏序集形式的容斥原理\\
对于一个偏序集$ (\mathcal{P}(X_n),\subseteq) $, ($X_n$为$n$元集) ,若
$$
F,G:\mathcal{P}(X_n)\rightarrow R
$$
且
$$
G(K)=\sum_{L\subseteq K} F(L)\qquad (K\subseteq X_n)
$$
考虑反解,有:
$$
F(K)=\sum_{L\subseteq K} (-1)^{|K|-|L|}G(L)
$$
\begin{proof}
$$
\begin{aligned}
\sum_{L\subseteq K} (-1)^{|K|-|L|} G(L)&= \sum_{L\subseteq K} (-1)^{|K|-|L|} \sum_{T\subseteq L} F(T)\\
&= \sum_{T\subseteq K} F(T) \sum_{T\subseteq L\subseteq K} (-1)^{|K|-|L|}\\
&=F(K)
\end{aligned}
$$
\end{proof}
这就是莫比乌斯反演.\\
因此我们可以对$F,G$下定义,令$A_1,A_2,\cdots,A_n$是有限集$S$的子集,且$K\subseteq X_n$,$F(K)$为恰好属于所有$A_i$ that $i\notin K$的元素
个数,即
$$
F(K)=\left| \bigcap_{i\notin K} A_i -\bigcup_{i \in K} A_i \right|
$$
显然有
$$
F(X_n)=n-\left| \bigcup_{i \in X_n} A_i  \right|
$$
然后令
$$
G(K) =\sum_{L\subseteq K} F(L)=\left| \bigcap_{i \notin K} A_i \right|
$$
由莫比乌斯反演有
$$
F(K)=\sum_{L\subseteq K} (-1)^{|K|-|L|}G(L)
$$
有
$$
|\overline{A_1}\cap \overline{A_2} \cap \cdots \cap \overline{A_n}|=\sum_{J\subseteq K} (-1)^{|J|} \left|\bigcap_{i\in J}A_i \right|
$$
等价于上面的容斥原理.

\subsubsection{偏序集里的莫比乌斯反演}
建议先看看代数系统.
%%%~~~~~~~~~~~~~~~~~~~~~~~~~~~~~~~~~~~~~~~~~~~~~~~~~~~~~~~~~~~~~~~
下面将莫比乌斯反演推广到偏序集$(X,\le)$ 里.以下介绍的函数满足
$$
f:X \times X \rightarrow \mathcal{R}
$$
且$f(x,y)=0$ if $x\nleq y$.
下面考察代数系统$<\mathcal{F},*>$,设其为A
\begin{definition}
令$h=f*g$为$f$和$g$的卷积,如果满足:
$$
h(x,y)=\begin{cases} \sum_{z:x\leq z\leq y} f(x,z)g(z,y) \qquad&,x\leq y\\ 0\qquad &,other\end{cases}
$$
\end{definition}
显然卷积运算在该偏序集上是封闭的,故这是一个广群.
并且显然其是满足 \textbf{结合律} ,故其是一个半群.
\begin{definition}
科罗内尔delta函数:
$$
\delta(x,y)=\begin{cases}1\qquad ,x=y\\ 0\qquad,other\end{cases}
$$
\end{definition}
显然有$f*\delta=\delta*f=f$,显然其为\textbf{卷积运算的幺元}.
故这个一个独异点.

\begin{definition}
$\zeta$函数:
$$
\zeta(x,y)=\begin{cases}1\qquad,x\leq y\\ 0\qquad,other\end{cases}
$$
\end{definition}


\begin{definition}
逆函数:\\
对于$X$中所有的$y$满足$f(y,y)\neq 0$,有其逆元.
$$
g(x,y)=\begin{cases}
  \frac{1}{f(y,y)} ,& x=y\\
  -\frac{1}{f(y,y)}\sum_{x\le z < y} g(x,z)f(z,y),&x< y\\
  0,&other
\end{cases}
$$
\end{definition}


\begin{proof}
  若$x\neq y$
  $$
  \begin{aligned}
  (g*f)(x,y)&=g(x,y)f(y,y)+\sum_{x \le z <y} g(x,z)f(z,y)\\
  &=-\sum_{x \le z <y} g(x,z)f(z,y)+\sum_{x \le z <y} g(x,z)f(z,y)\\
  &=0
  \end{aligned}
  $$
  故g是其左逆元,类似地可以证明其是右逆元.故其是f的逆元.
\end{proof}


\begin{definition}
  莫比乌斯函数:\\
  莫比乌斯函数为 $\zeta$函数的逆函数.\\
  具体地:
  $$
  \mu(x,y)=\begin{cases}
    1&,x=y\\
    -\sum_{x\leq z<y} \mu(x,z)& ,x<y
    \end{cases}
  $$
\end{definition}

\newp{3}

下面给出一些常见偏序集的莫比乌斯函数: 

1.$(\mathcal{P}(X_n),\subseteq)$ \\
\[
\mu(A,B)=(-1)^{|B|-||A}
\]

2.$(X_n,\le)$ 即正整数集合上的全序关系\\
\[
\mu(k,l)=\begin{cases}
  1&\qquad,l=k\\
  -1&\qquad,l=k+1\\
  0&\qquad,other
  \end{cases}
\]
3.$(X_n,|)$,即正整数集合上的整除关系\\
有$\mu(a,b)=\mu\left(1,\frac{b}{a}\right)$
\[
  \mu(1,n)=\begin{cases}
    1&\qquad,n=1\\
    (-1)^k&\qquad,\text{n是互不相同的素数乘积}\\
    0,&\qquad,other
    \end{cases}
\]

4.直积的莫比乌斯函数\\
线性有限偏序集$(X,\le_1)$,$(Y,\le_2)$,且$\mu_1$,$\mu_2$分别为其莫比乌斯函数,定义其笛卡尔积的偏序为
$$
(x,y)\le (x',y') \iff x\le x' \text{and} y \le y'
$$
那么新偏序集$(X\times Y ,\le_3 )$的莫比乌斯函数为
$$
\mu((x,y),(x',y'))=\mu_1(x,x')\mu_2(y,y')
$$


\begin{theorem}
  莫比乌斯反演:\\
设$(X,\le)$是一个具有最小元的线性偏序集.令$\mu$ 是其莫比乌斯函数,定义在$X$上的实值函数$F,G:X\rightarrow \mathcal{R}$满足
$$
G(x)=\sum_{z\le x} F(z),\qquad (x\in X)
$$
那么有
$$
F(x)=\sum_{y\le x} \mu(y,x) G(y),\qquad (x\in X)
$$
\end{theorem}

\begin{proof}
  $$
  \begin{aligned}
  \sum_{y \le x} \mu(y,x) G(y) &=\sum_{y \le x} \mu(y,x) \sum_{z\le y} F(z)\\
  &=\sum_{z \le x}F(z) \sum_{ z\le y \le x} \zeta(z,y)\mu(y,x)\\
  &=\sum_{z\le x} F(z) \delta(z,x)\\
  &=F(x)
  \end{aligned}
  $$
\end{proof}
这里最小元保证了和式有限,因此不用判断敛散性.(这里对和式的一些变换在无穷和式有的有时不成立)\\ 
事实上,莫比乌斯反演是卷积结合律的一个推论.\\
\begin{proof}
  不妨设最小元为0,定义$f,g\in \mathcal{F}(X)$
  $$
  f(x,y)=\begin{cases}
  F(y)&, x=0\\
  0&,other
  \end{cases}
  $$

  $$
  g(x,y)=\begin{cases}
  G(y)&, x=0\\
  0&,other
  \end{cases}
  $$
  从而有$g=f*\zeta$,从而有$g*\mu=f$,

\end{proof}
%但若是满足$\forall x \in X ,\sum_{z\le x} F(z) ,\sum_{z \le x} G(z)$绝对收敛,就不再需要最小元这一条件.\\


\newp{3}
\section{递推关系和生成函数}
一些斐波拉契数列的性质:\\
1.
$$
\begin{pmatrix}
  F_n &F_{n+1}
\end{pmatrix}=
\begin{pmatrix}
  F_0 &F_1
\end{pmatrix}
\begin{pmatrix}
  0 &1\\1&1
\end{pmatrix}^n
$$
2.
$F_{2k}=F_k(2F_{k+1}-F_k)$; 
$F_{2k+1}=F_{k+1}^2+F_{k}^2$

3.
$$
\sum_{i=0}^{n} f_i=f_{n+2}-1
$$

4.
$$
2|f_n\iff3|n
$$

\subsection{生成函数}
这里只做简单介绍\\
牛顿二项式定理
\begin{theorem}
  设$\alpha$ 是一个实数.对于任意$x,y$ with $0\le |x| <|y|$,有性质
  $$
  (x+y)^{\alpha} =\sum_{k=0}^{\infty} \binom{\alpha}{k} x^{k} y^{\alpha-k}
  $$
  where
  $$
  \binom{\alpha} {k}=\frac{\alpha (\alpha-1) \cdots (\alpha -k+1)}{k!}
  $$
\end{theorem}
设$|z|<1$,特别地有
$$
(1+z)^{\alpha} =\sum_{k=0}^{\infty} \binom{\alpha}{k} z^k
$$

\begin{proposition}
  若$\alpha$是一个负整数,且$\alpha = -n$then
  $$
  \begin{aligned}
    \binom{\alpha}{k}&=\binom{-n}{k}\\
    &=\frac{-n (-n-1)\cdots (-n-k+1)}{k!}\\
    &=(-1)^k\binom{n+k-1}{k}
  \end{aligned}
  $$
\end{proposition}
thus: for $|z|<1$
$$
(1+z)^{-n}=\frac{1}{(1+z)^n}=\sum_{k=0}^{\infty}(-1)^k \binom{n+k-1}{k} z^k
$$

\subsubsection{一般生成函数}
无穷数列$h_0,h_1,\cdots $的生成函数为$g(x)=h_0+h_1 x+h_2x^2+\cdots $
\\

生成函数的一些性质\\
设$H$为数列,$F$为其对应的生成函数\\
1. $cH \rightarrow cF$\\
2. $H_1+H_2 \rightarrow F_1 +F_2$\\
3.${0,0,0,\cdots}+H \rightarrow x^kF$\\
4. $iH(H_1,2H_2,\cdots) \rightarrow F'$\\
5. 令$G_n=\sum_{i+j=n} H_{1i}\cdot H_{2j}$
那么$G\rightarrow F_1\cdot F_2$\\
一面介绍两种重要的生成函数即:多重集合的$n$组合级数的生成函数\\
根据泰勒级数
\[
\frac{1}{1-x}=\sum_{n=0}^{\infty}x^n
\]
我们可以解$h_n$表示
\[
e_1+e_2+\cdots+e_k=n
\]
的非负整数解的个数。

其生成函数为
$$
g(x)=\sum_{n=0}^{\infty} \binom{n+k-1}{n}x^n=\frac{1}{(1-x)^k}
$$

\begin{example}
  设$x_1+x_2+x_3+x_4=n$的整数解个数,其中$x_1$是偶数,$x_2$是5的倍数,$x_3\le 4$,$x_4\le 1$ \\
  解:\\
  $$
  \begin{aligned}
    g(x)&=(1+x^2+x^4+\cdots)(1+x^5+\cdots)(1+x+x^2+x^3+x^4)(1+x)\\
    &=\frac{1}{1-x^2} \frac{1}{1-x^5} \frac{1-x^5}{1-x} (1+x)\\
    &=\frac{1}{(1-x)^2}\\
    &=\sum_{n=0}^{\infty} \binom{n+1}{n} x^n
  \end{aligned}
  $$
  故为n+1.
\end{example}
我们得到几个小结论:\\
1.限制$\ge k$,可以乘$x^k$\\
2.限制$\le k$,少写几项\\
3.是$k$的倍数,整体代换\\

\subsubsection{指数生成函数}
无穷数列$h_0,h_1,\cdots $的指数生成函数为$g(x)=h_0+h_1 \frac{x}{1!}+h_2 \frac{x^2}{2!}+\cdots $\\
下面给出一类常用的指数生成函数,即多重集合的n排列数的生成函数.\\
\begin{theorem}
  设$S$ 是多重集合$\{n_1 a_1\,\cdots n_k a_k\}$,其中$n_i\ge 0$,那么数列的指数生成函数为\\
  $$
  g(x)=f_{n_1}(x)f_{n_2}(x) \cdots f_{n_k} (x)
  $$
  其中
  $$
  f_{n_i}(x)=\sum_{k=0}^{n_i} \frac{x^k}{k!}
  $$
\end{theorem} 

\begin{example}
  用红,白,蓝,绿色给$1\times n$棋盘染色,其中要求红色为偶数,白色是奇数,求方案数\\
  解:\\
  $$
  \begin{aligned}
    g(x)&=\left(\sum_{n=0}^{\infty}\frac{x^n}{n!} \right)^2\left(1+\frac{x^2}{2!}+\cdots \right)\left(x+\frac{x^3}{3!}+\cdots \right)\\
    &=e^{2x}\left(\frac{e^x+e^{-x}}{2} \right) \left( \frac{e^x-e^{-x}}{2} \right)\\
    &=\frac{e^{4x}-1}{4}\\
    &=\frac{1}{4}\sum_{n=0}^{\infty} 4^{n} \frac{x^n}{n!}-\frac{1}{4}\\
    &=\sum_{n=1}^{\infty} 4^{n-1} \frac{x^n}{n!}
    \end{aligned}
  $$
  故为$4^{n-1}$
\end{example}
1.对于偶数限制此项为\\
\[
\frac{e^x-e^{-x}}{2}
\]
2.奇数限制\\
\[
\frac{e^x+e^{-x}}{2}
\]
对于求解线性齐次递推关系这里不做介绍


\section{卡特兰数和第二类斯特林数}
\subsection{卡特兰数}
折线图\\
只有两类线段(a,b)-(a+1,b+1)或(a,b)-(a+1,b-1)\\
\begin{proposition}
  $A_0(a_0,b_0)$,$A_n(a_n,b_n)$能用折线连接的充要条件是\\
  $|b_n-b_0|\le a_n-a_0=n$且$2|(|b_n-b_0|+n)$\\
  连接这两点的折线有
  $$
  \binom{n}{\frac{n+b_n-b_0}{2}}
  $$
  条.
\end{proposition}
卡特兰数:\\
$$
C_n=\frac{1}{n+1}\binom{2n}{n}=\binom{2n}{n}-\binom{2n}{n-1}
$$
递推式\\
$$
C_n=\sum_{k=1}^{n}C_{k-1}C_{n-k}=\frac{1}{n+1} (4n-2)C_{n-1}
$$
Catalan 数列$C_n$可以应用于以下问题:\\


\subsection{第二类斯特林数}

%下一章节~~~~~~~~~~~~~~~~~~~~~~~~~~~~~~~~~~~~~~~~~~~~~~~~~~~~~~~~~~~~~~~~~~~~~~~~~~~~~~~~~
\chapter{数论}
\section{整除}

\begin{proposition}
  令$a,b,c$为整数,那么有:
  $$
  \gcd(a+cb,b)=\gcd(a,b)
  $$
\end{proposition}

\begin{definition}
  $a,b,m,n\in \mathcal{Z}$,称$ma+nb$为$a,b$的线性组合
\end{definition}


\begin{theorem}
  裴蜀定理:\\
  如果$a,b$均为整数,则有整数$m$和$n$,使得
  $$
  ma+nb=\gcd(a,b)
  $$
\end{theorem}
其中该等式又被称为裴蜀等式,$m,n$被称为裴蜀数.\\
可以用扩展欧几里得算法求出$ma+nb=\gcd(a,b)$的特解,然后有通解
$$
\begin{cases}
  m&=m_0+k\frac{b}{\gcd(a,b)}\\
  n&=n_0-k\frac{a}{\gcd(a,b)}
\end{cases}
$$
注意到
$$
a(m+bu)+b(n-au)=\gcd(a,b)
$$
故满足等式的$m,n$有无穷多对.
\begin{lemma}
  两个不全为0的整数$a,b$的最大公因数是其线性组合中最小的正整数.
\end{lemma}

\begin{proof}
  不妨设$d$是$a,b$线性组合中最小的正整数.
  考虑带余除法:\\
  $a=dq+r$从而有
  $r=a-dq=a-q(ma+nb)=(1-qm)a-qnb$\\
  因此$d|a$,同理$d|b$,故$d$为公因数.\\
  不妨设$e=\gcd(a,b)$,那么$d|e$,又$e|(ma+nb)$,即$e|d$\\
  故$e=d$
\end{proof}

\begin{theorem}
  如果$a,b$是整数,那么所有 $a,b$ 的线性组合所构成的集合与所有
  $\gcd(a,b)$  的倍数所构成的集合相同.
  换言之,所有 $a,b$ 的线性组合,都是 $\gcd(a,b)$ 的倍数.
\end{theorem}

\begin{theorem}
  如果 $a,b$ 是不全为0的整数,那么正整数 $d$ 是 $a,b$的最大公因数,当且仅当
  1.$d|a,d|b$
  2.若$c|a,c|b$,那么$c|d$
\end{theorem}

\begin{definition}
  令 $a_1,a_2,\cdots ,a_n$为不全为0的整数,如果 $d$
 为他们公因子中最大的一个,则称 $d$
 为 $a_1,a_2,\cdots ,a_n$ 的最大公因数.
记为
  $ \gcd(a_1,a_2,\cdots ,a_n)=d$

\end{definition}

\begin{theorem}
$$
\gcd(a_1,a_2,\cdots ,a_n)=\gcd(a_1,a_2,\cdots, \gcd(a_{k-1},a_k),\cdots a_n)
$$
\end{theorem}


\begin{definition}
  我们称 $a_1,a_2,\cdots ,a_n $ 互素 如果 $ \gcd(a_1,a_2,\cdots ,a_n)=1$
\end{definition}

\begin{definition}
  我们称 $a_1,a_2,\cdots ,a_n $两两互素,如果任意两个数互素
\end{definition}

\begin{theorem}
  若$\gcd(a,m)=1$,$\gcd(b,m)=1$,则$\gcd(ab,m)=1$\\
  若$\gcd(a,b)=1$,则$\gcd(a^k,b^l)=1$
\end{theorem}

\begin{theorem}
  设正整数$a,b$之积 是一个整数的 $k(k\ge 2)$ 次幂.若
  $\gcd(a,b)=1$. 则 $a,b$都是整数的  $k$次幂.
  一般地:设正整数 $ a_1,a_2,\cdots ,a_n $ 之积是一个正整数的 $k$ 次幂.若 
  $a_1,a_2,\cdots ,a_n$ 两两互素,则 $a_1,a_2,\cdots ,a_n$
 都是整数的 $k$
 次幂.
\end{theorem}

\begin{lemma}
$$
\gcd(a_1^k,a_2^k,\cdots ,a_n^k)=gcd^k(a_1,a_2,\cdots ,a_n)
$$
\end{lemma}


\begin{corollary}
  裴蜀定理可以推广到 $n$ 个整数的情形:设 $a_1, a_2, \dots, a_n$
   是不全为零的整数,则存在整数 $x_1, x_2, \dots, x_n$, 使得 $a_1 x_1 + a_2 x_2 + \cdots + a_n x_n=\gcd(a_1, a_2, \dots, a_n)$。
   其逆定理也成立:设 $a_1, a_2, \dots, a_n$ 是不全为零的整数,$d > 0$ 是 $a_1, a_2, \dots, a_n$ 的公因数,
   若存在整数 $x_1, x_2, \dots, x_n$, 使得 $a_1 x_1 + a_2 x_2 + \cdots + a_n x_n=d$,则 $d = \gcd(a_1, a_2, \dots, a_n)$。
\end{corollary}

\begin{corollary}
  对自然数 $a、b$ 和整数$ n,a$ 与 $b $互素,考察不定方程:$ax+by=n$
 其中 $x$ 和 $y$ 为自然数。如果方程有解,称 $n$ 可以被 $a、b$ 表示。
 记 $C=ab-a-b$。由 $a$ 与 $b$ 互素,$C$ 必然为奇数。则有结论:
对任意的整数 $n$,$n$ 与 $C-n$ 中有且仅有一个可以被表示。
即:可表示的数与不可表示的数在区间 $[0,C]$ 对称(关于 $C$ 的一半对称)。$0$ 可被表示,$C$ 不可被表示;负数不可被表示,大于 $C$ 的数可被表示。
\end{corollary}

\begin{corollary}
  二元一次不定方程有非负整数解的条件\\
  $a,b>0$,若$ax+by=n$,$(a,b)=1$,则$n>ab-a-b$时有解,解的个数为$\left\lfloor \frac{n}{ab} \right\rfloor$
  $\left\lfloor \frac{n}{ab} \right\rfloor+1$ 
\end{corollary}

\begin{proposition}
  $$
  \gcd(a_1,a_2,\cdots,a_n)=\gcd(a_1,a_2-a_1,\cdots,a_n-a_{n-1})
  $$
\end{proposition}

\newp{3}
一些小结论\\
  1.在[1e18]的范围下,一个数最多与连续7个数不互质.\\
  2.一个数能被4整除,当且仅当末尾两位能被4整除\\
  3.一个数能被25整除,当且仅当末尾两位能被25整除\\
  4. 一个数能被8整除,当且仅当末尾三位能被8整除\\
  5.一个数能被125整除,当且仅当末尾三位能被125整除\\
  6. 一个数能被3整除,当且仅当各位数之和能被3整除\\
  7. 一个数能被9整除,当且仅当各位数之和能被9整除\\
  8. 能被7整除的数的特征:a.抹去个位数b.减去原个位数的2倍c.其差能被7整除。\\
  9. 能被11整除的数的特征: a.抹去个位数 b.减去原个位数 c.其差能被11整除。或:奇数位上的数字和与偶数位上的数和相减,其差能被11整除
\section{同余}
以下所有参数未特殊说明,均为默认整数,模数默认正整数\\

\subsection{同余}

\begin{definition}
  设$m$是正整数,若$m|(a-b)$,则称$a$和$b$模$m$同余.记作$a \equiv b(\mod m)$
\end{definition}

性质\\
1. $a \equiv b (\mod m) \iff \exists k (k\in Z),a=b+kz$\\
2. $a \equiv a(\mod m)$\\
3. $a \equiv b(\mod m) \rightarrow b \equiv a(\mod m)$\\
4. $a \equiv b(\mod m),b \equiv c(\mod m) \rightarrow a \equiv c(\mod m)$\\
5. $a \equiv b(\mod m),c \equiv d(\mod m)\rightarrow a+c \equiv b+d(\mod m) ,a-c \equiv b-d(\mod m),ac \equiv bd(\mod m)$\\
6. $a^n \equiv b^n(\mod m)$\\
7. $ac \equiv bc (\mod m),d=\gcd(c,m)\rightarrow a \equiv b (\mod m/d)$\\
8. $a \equiv b (\mod m),n|m \rightarrow a \equiv b (\mod n)$\\
9. $a \equiv b (\mod m),a \equiv b (\mod n)\rightarrow a \equiv b (\mod lcm(m,n))$\\


\begin{definition}
  设模为$n$,则根据余数可将所有的整数分为$n$类,
  把所有与整数$a$模$n$同余的整数构成的集合叫做模$n$的一个剩余类,
  记作$[a]$。并把$a$叫作剩余类$[a]$的一个代表元。
\end{definition}

\begin{definition}
  从模$n$的每个剩余类中各取一个数,得到一个由$n$个数组成的集合,叫做模$n$的一个完全剩余系。
\end{definition}

\begin{proposition}
  若$r_1,r_2,\dots ,r_m$ 是模$m$的一个完全剩余系,且正整数$a$满足$\gcd(a,m)=1$,则对任何整数$b$,$ar_i+b$也为一个完全剩余类.
\end{proposition}

\begin{proof}
  若不然,则存在$ar_i+b \equiv ar_j+b (\mod m) \iff ar_i \equiv ar_j (\mod m) \iff m|a(r_i-r_j) \iff m|a_i-a_j \iff r_i \equiv r_j (\mod m)$与条件矛盾.故得证
\end{proof}


\begin{proposition}
  $a_1,a_2,\dots,a_m$ 是模$m$的一个完全剩余系,$b_1,b_2,\dots b_n$ 是模$n$ 的一个完全剩余系, 
  且$\gcd(n,m)=1$那么$na_i+mb_j$ 是模$mn$ 的一个完全剩余系
\end{proposition}

\begin{proof}
  首先由乘法原理知道有$mn$个数,那么只需证两两不同余即可\\
若不然则对于$(a,b)\neq(a',b')$
$na+mb \equiv na'+mb' (\mod mn) \iff mn|n(a-a')+m(b-b') \iff
m|(a-a'),n|(b-b') \iff a \equiv a' (\mod m),b \equiv b' (\mod n)$ 与条件矛盾.故得证.
\end{proof}

\begin{definition}
  简化剩余系也称既约剩余系或缩系,是$m$的完全剩余系中与$m$互素的数构成的子集,
  如果模$m$的一个剩余类里所有数都与$m$互素,就把它叫做与模$m$互素的剩余类。在与模$m$互素的全体剩余类中,
  从每一个类中各任取一个数作为代表组成的集合,叫做模$m$的一个简化剩余系。
\end{definition}

\begin{proposition}
  若$r_1,r_2,\dots ,r_m$ 是模$m$的一个缩系,且正整数$a$满足$\gcd(a,m)=1$,则$ar_i$也为一个缩系.
\end{proposition}


\begin{proof}
  由完系性质$1$可知其两两不同余,故只需证明其与均$m$互质即可.
因为$gcd(r_i,m)=1,gcd(a,m)=1\Rightarrow gcd(ar_i,m)=1$得证
\end{proof}

\begin{proposition}
  $a_1,a_2,\dots,a_m$ 是模$m$的一个缩系,$b_1,b_2,\dots b_n$ 是模$n$ 的一个缩系, 且$gcd(n,m)=1$那么$na_i+mb_j$ 是模$mn$ 的一个缩系
\end{proposition}

\begin{proof}
  只需证明其是所有与$mn$互质的剩余类.
$$
\begin{aligned}
  &\gcd(a_i,m)=1,\gcd(b_j,n)=1,\gcd(n,m)=1 \\&\Rightarrow \gcd(na_i,m)=1,\gcd(mb_j,n)=1\\  &\Rightarrow \gcd(na_i+mb_j,n)=1,\gcd(na_i+mb_j,m)=1\\& \Rightarrow \gcd(na_i+mb_j,mn)=1
\end{aligned}
$$
若将$a_i,b_j$扩展成完系,若$\gcd(na_i+mb_j,mn)=1 \Rightarrow \gcd(na_i+mb_j,m)=1,\gcd(na_i+mb_j,n)=1\Rightarrow \dots$ 逆着证回去即可.
\end{proof}


\subsection{线性同余方程}
\begin{definition}
  形如$ax \equiv b (\mod m)$ 的同余式称为一元线性同余方程
\end{definition}

\begin{theorem}
  $\gcd(a,m)=d,d\nmid b$,则无解,否则恰好有$d$个模$m$不同余的解.
\end{theorem}


\begin{proof}
  若$d\nmid b$,则 $ax \equiv b (\mod m)\iff ax-ym=b$,根据贝祖定理显然无解.\\
若$d\mid b$,则显然有无穷多组解,我们设其中一组特解为$x_0,y_0$ \\
其通解为$x=x_0+(m/d)t,y=y_0+(a/d)t$\\
设$x_1=x_0+(m/d)t_1,x_2=x_0+(m/d)t_2$\\
$x_1 \equiv x_2 (\mod m) \iff t_1 \equiv t_2 (\mod d)$\\
所以有$d$ 个不同于的解.
\end{proof}


\begin{definition}
  $\gcd(a,m)=1,ax \equiv 1 (\mod m)$ 则称该同余方程的一个解为$a$模$m$的逆,记为$a^{-1}$.
显然$\gcd(a^{-1},m)=1$
\end{definition}

\begin{theorem}
  设$p$ 为素数,正整数$a=a^{-1}$ ,当且仅当$a \equiv \pm 1(\mod p)$.
\end{theorem}


\begin{proof}
$a \equiv \pm 1(\mod p) \iff a^{2} \equiv 1(\mod p)$
反过来.有\\
$a^{2} \equiv 1(\mod p) \Rightarrow p|(a^2-1) \Rightarrow p|(a+1) 或p|(a-1) \Rightarrow a \equiv \pm 1(\mod p)$
\end{proof}

\begin{theorem}
  威尔逊定理\\
  若$p$是素数,则$(p-1)! \equiv -1 (\mod p)$
\end{theorem}

\begin{proof}
  $p=2$ 显然成立.\\
否则对于$1 \leq a \leq p-1$,可以找到其逆元与之配对,且除$1$和$p-1$都能两两配对.\\
故$(p-1)! \equiv p-1 \equiv -1 (\mod p)$
\end{proof}

\begin{theorem}
  威尔逊定理逆定理\\
  若$n\geq 2$是正整数,且$(n-1)! \equiv -1 (\mod n)$则 $n$为质数
\end{theorem}

\begin{proof}
  若不然,设$n$ 为合数,则其必存在小于$n$ 的素因子$p$\\
所以有$(n-1)! \equiv -1 (\mod n),p|n \Rightarrow (n-1)! \equiv -1 (\mod p)$,\\
但是 $(n-1)! \equiv 0 (\mod p)$\\
而$n>1$ 矛盾.故得证.
\end{proof}

\begin{theorem}
  费马小定理\\
  如果$p$是一个素数,$a$是正整数且$a$不是$p$的倍数,则$a^{p-1} \equiv 1 \pmod{p}$。
\end{theorem}

\begin{proof}
  因为$\gcd(a,p)=1$\\
所以 $\prod_{i=1}^{p-1} ia \equiv \prod_{i=1}^{p-1} i \pmod{p} \Rightarrow a^{p-1} \equiv 1 \pmod{p}$ 得证.
\end{proof}

\begin{definition}
  欧拉函数\\
  对于正整数 $n$,小于等于 $n$ 且与 $n$ 互质的正整数的个数,称为欧拉函数,记作$\phi{(n)}$
\end{definition}

\begin{theorem}
  设$m$ 是一个正整数,$a$ 是一个正整数且$\gcd(a,m)=1,那么a^{\phi(m)} \equiv 1 \pmod{m}$
\end{theorem}

\begin{proof}
  设$r_1,r_2,\dots r_{\phi(m)}$,是不超过$m$ 的模$m$的一个缩系.\\
那么$ar_1,ar_2,\dots ar_{\phi(m)}$也是一个缩系\\
故$ar_1ar_2\dots ar_n \equiv r_1r_2\dots r_n \pmod{m} \iff a^{\phi(m)} \equiv 1 \pmod{m}$\\
得证.
\end{proof}

\begin{definition}
  同余方程组是指一组形如下面的方程的集合:
$$
\begin{cases}
a_1 &\equiv b_1 \pmod{m_1} \\
a_2 &\equiv b_2 \pmod{m_2} \\
\vdots \\
a_n &\equiv b_n \pmod{m_n}
\end{cases}
$$

其中,$a_i$和$b_i$是整数,$m_i$是正整数。
这组方程要求对于每个$i$,$a_i$除以$m_i$的余数等于$b_i$除以$m_i$的余数,即$a_i$与$b_i$在模$m_i$下同余。
解同余方程组就是要找到满足所有这些条件的整数解。
\end{definition}

\begin{example}
  $$
\begin{cases}
x \equiv 1 \pmod{3} \\
x \equiv 2 \pmod{5} \\
\vdots \\
x \equiv 3 \pmod{7}
\end{cases}
$$

我们可以使用迭代法(逐级满足法)解决.
由第一个式子得 $x=3t+1$
然后带入$3t+1 \equiv 2 \pmod{5} \iff t \equiv 4\pmod{5}$
以此类推.
但是这只能解决一些简单的问题,下面我们给出一般地解法.
\end{example}


\begin{theorem}
  中国剩余定理(CRT)\\
  设$m_1,m_2,\dots ,m_r$ 是两两互素的正整数,则同余方程
$$
\begin{cases}
x \equiv a_1 \pmod{m_1} \\
x \equiv a_2 \pmod{m_2} \\
\vdots \\
x \equiv a_r \pmod{m_r}
\end{cases}
$$
有模$m_1m_2\dots m_r$ 的唯一解
$$
x=\sum_{i=1}^{r} a_iM_iM_i^{-1} 
$$
其中
$$
M_i= \frac{1}{m_i}\prod_{j=1}^r m_j ,\qquad M_iM_i^{-1} \equiv 1 \pmod{m_i}
$$
\end{theorem}

\begin{proof}
  先证明$x$是方程组的解.\\
对于任意一个方程有,\\
$x \equiv a_kM_kM_k^{-1}  \equiv a_k \pmod{m_k}$ ,显然成立.\\
下证唯一性.\\
若$x_1,x_2$ 为方程组的2个解,则有$x_1 \equiv x_2 \pmod{m_k} \iff m_k|(x_1-x_2) \iff M|(x_1-x_2) \iff x1 \equiv x2 \pmod{M}$
\end{proof}


\begin{theorem}
  拉格朗日定理\\
  $p$为素数,$f(x)=a_nx^n+a_{n-1}x^{n-1}+\dots +a_1x+a_0(p \nmid a_n)$ 
  是模$p$意义下的整系数多项式方程,则同余方程$f(x) \equiv 0 \pmod{p}$ 在模$p$ 意义下至多有$n$ 个不同的解.
\end{theorem}


\begin{corollary}
  若超过$n$个解,则$p \mid a_i(i=0,1,\dots n)$,即$f(x) \text{是模p意义下的零多项式}$
\end{corollary}

\begin{corollary}
  若$n \leq p$ 则同余式$f(x)=x^n+a_{n-1}x^{n-1}+\dots +a_1x+a_0 \equiv 0 \pmod{p}$有$n$ 
  个解的充要条件是$x^p-x$ 除以$f(x)$所得的余式的一切系数都是$p$ 的倍数
\end{corollary}
这里介绍一个比较重要的多项式,常用于构造\\
$$
f(x)=\prod_{i=1}^{p-1}(x-i) -(x^{p-1}-1)
$$ 

\begin{theorem}
  wolstenholme定理\\
  若$p$为大于3的素数,则
  $$
  \sum_{k=1}^{p-1} \frac{(p-1)!}{k} \equiv 0 \pmod{p^2} 
  $$
\end{theorem}



\section{乘性函数}

\begin{definition}
  算术函数\\
  定义在所有正整数上的函数称为算数函数.
\end{definition}

\begin{definition}
  乘性函数\\
  若$\gcd(m,n)=1$,均有 $f(mn)=f(m)f(n)$,则称$f$ 为乘性函数.
\end{definition}

\begin{proposition}
  若$f$为乘性函数,$n=\prod_{i=1}^{k} p_i^{a_i}$ 为一个素因数分解.则$f(n)=\prod_{i=1}^k f(p_i^{a_i})$
由定义显然成立.
\end{proposition}

\begin{definition}
  和函数\\
  $f$ 为一个算术函数,$F(n)=\sum_{d \mid n} f(d)$ 称为$f$ 的和函数
\end{definition}

\begin{definition}
  欧拉函数\\
  $\phi(n)=\sum_{i=1}^n [gcd(i,n)=1]$ ,称为欧拉函数.
\end{definition}

\begin{proposition}
  设$p$是素数,$\phi(p^a)=p^a-p^{a-1} $
\end{proposition}

\begin{proof}
  由定义,$\phi(p^a)= \sum_{i=1}^{p^a} [gcd(i,p^a)=1]= \sum_{i=1}^{p^a} 1-[gcd(i,p^a) \neq 1]=p^a- \sum_{i=1}^{p^a} [gcd(i,p^a) \neq 1]$,
  这样的$i$ 显然只有$p$的倍数,有$p^{a-1}$ 个,证毕.
\end{proof}

\begin{proposition}
  欧拉函数是乘性函数
\end{proposition}

\begin{proof}
  若$gcd(m,n)=1$
由缩系的定义知道,显然模$m$的缩系有$\phi(m)$个数,模$n$的缩系有$\phi(n)$个数,由缩系的一个性质知模$mn$ 的缩系有$\phi(m)\phi(n)$个数.故$\phi(mn)=\phi(m)\phi(n)$
\end{proof}

\begin{proposition}
  $n>2$,$\phi(n)$ 为偶数
\end{proposition}

\begin{proposition}
 $$
 \phi(n)=n\prod_{i=1}^k(1-\frac{1}{p_i})
 $$
\end{proposition}

\begin{proof}
  容斥原理可证明
这里用乘性函数的性质证明
$$
\begin{aligned}
  \phi(n)&=\prod_{i=1}^k \phi(p_i^{a_i})\\
  &=\prod_{i=1}^k(p^{a_i}-p^{a_i-1})\\
  &=\prod_{i=1}^{k} p_i^{a_i}\prod_{i=1}^k(1-\frac{1}{p_i})\\
  &= n\prod_{i=1}^k(1-\frac{1}{p_i})
\end{aligned}
$$
\end{proof}


\begin{proposition}
  欧拉函数的和函数
  $$
  F(n)=\sum_{d \mid n} \phi(d)=n
  $$
\end{proposition}

\begin{proposition}
  定义$C_d$ 为$1$到$n$中与$n$最大公因数为$d$ 的集合容易证明其是$1$到$n$构成的正整数集合的一个划分.
而$C_d$中有$\phi(n/d) $ 个元素,(若$a\in C_d$ 则$ gcd(a/d,n/d)=1 $),故$n=\sum_{d|n}C_d=\sum_{d|n}\phi(n/d)=\sum_{d|n}\phi(d)$ 得证.
\end{proposition}

\begin{definition}
  狄利克雷卷积\\
  $f$,$g$ 为算数函数,定义狄利克雷积为
  $$
  (f*g)(n)=\sum_{d|n}f(d)g\left(\frac{n}{d} \right)
  $$
\end{definition}
实际上这整数上一章的卷积在整除关系上的定义\\


性质\\
1. $f*g=g*f$\\
2. $(f*g)*h=f*(g*h) $\\
3. $f*(g+h)=f*g+f*h $\\

这里的算数函数就是$f(1,x)\in \mathcal{F}$\\
故有逆元的条件是$f(1)\neq 0$\\

\begin{theorem}
  如果$f$,$g$ 是乘性函数,则$f*g$也是乘性函数
\end{theorem}


\begin{theorem}
  若$F=f*g$,$h$ 是$g$ 的逆函数,那么$f=F*h$
\end{theorem}

\begin{theorem}
  乘性函数的和函数也是乘性函数
\end{theorem}

\begin{definition}
  因子和与因子个数函数\\
  $\sigma(n)=\sum_{d|n} d $\\
  $\tau(n)=\sum_{d|n}1 $
\end{definition}

\begin{proposition}
  因子和与因子个数函数均为乘性函数
\end{proposition}

\begin{proposition}
  设$n=\prod_{i=1}^{k} p_i^{a_i}$\\
  \[
    \begin{aligned}
      \sigma(n)&=\prod_{j=1}^{k} \frac{p_j^{a_j+1}-1}{p_j-1}\\
      \tau(n)&=\prod_{j=1}^{k} (a_j+1)
    \end{aligned}
  \]
\end{proposition}


\begin{definition}
  莫比乌斯函数\\
  \[
  \mu(n)=\begin{cases}
    1,& (n=1)\\
    (-1)^r,& (n=\prod_{i=1}^{r}p_i)\\
    0,& (other)
    \end{cases}
  \]
\end{definition}

\begin{theorem}
  $f=F*\mu$
\end{theorem}

\section{补充}
\subsection{勒让德公式}
\begin{definition}
  对于所有正整数 \(x \ge 1\) 和 \(k \ge 2\),定义 \(v_k(x!)\) 为 **在以 (k) 为底时,\(x!\) 的末尾零的个数**。
  形式化地说,\(v_k(x!)\) 是满足 \(k^i \mid x!\) 的最大整数 \(i\)。
\end{definition}

\begin{theorem}勒让德公式\\
  对于质数 \(p\),可以通过以下公式计算:
 \[
 v_p(x!) = \sum_{j=1}^\infty \left\lfloor \frac{x}{p^j} \right\rfloor
 \]
 其中 \(\lfloor y \rfloor\) 表示不超过 \(y\) 的最大整数(即**向下取整**)。\\
 当 \(k\) 不是质数时:

设其质因数分解为
$$
 k = \prod p_i^{e_i}
$$
 其中 \(p_i\) 是 \(k\) 的不同质因子,\(e_i\) 是相应的指数。

此时:
$$
 v_k(x!) = \min_i \left\lfloor \frac{v_{p_i}(x!)}{e_i} \right\rfloor
$$

\end{theorem}


\chapter{求和}
%该章节内容基本来自《具体数学》
一些记号

\textbf{调和数(harmonic numbe)}
$$
H_n=\sum_{k=1}^{n} \frac{1}{k}
$$

基本公式
$$
(x+y)^n=\sum_{k=0}^{n}\binom{n}{k}x^{n-k}y^{k}
$$

$$
a^n-b^n=(a-b)\sum_{1 \le k\le n} a^{n-k}b^{k-1}
$$

\section{递归问题 RECURRENT PROBLEMS}

\subsection{repertoire method}
例
$$
\begin{aligned}
f(1)&=\alpha\\
f(2n)&=2f(n)+\beta\\
f(2n+1)&=2f(n)+\gamma
\end{aligned}
$$
知
$$
f(n)=A(n)\alpha+B(n)\beta+C(n)\gamma
$$
通过对$f(n)$赋值或$(\alpha,\beta,\gamma)$赋值,求解。\\

在参数较少的情况下,可以将一些相同参数,分别设为独立参数,更容易找到有解的情况。
\subsection{约瑟夫问题}
形如
$$
\begin{aligned}
f(j)&=\alpha_j,\qquad &&1\le j<d\\
f(dn+j)&=cf(n)+\beta_j,\qquad  &&0\le j<d,n\ge 1
\end{aligned}
$$
有
$$
f((b_mb_{m-1}...b_0)_d)=(\alpha_{b_m}\beta_{b_{m-1}}...\beta_{b_0})_c
$$




\section{和式 SUMS} 

用
$$
\sum_{P(k)} a_k
$$
表示。

\subsection{和式和递归式 SUMS AND RECURRENCES}

和式可以表示为递归形式:
$$
\begin{aligned}
&S_0=a_0\\
&S_{n+1}=S_{n}+a_{n+1}
\end{aligned}
$$
用repertoire method解。\\
\begin{example}
  计算
$$
\sum_{k=0}^{n}(a+bn)
$$
写成递归式
$$
\begin{aligned}
&S_0=\alpha\\
&S_n=S_{n-1}+\beta n+\gamma
\end{aligned}
$$
其中
$$
\alpha=\gamma=a,\beta =b
$$
设
$$
S_n=A(n) \alpha +B(n) \beta+C(n) \gamma
$$
带入$1,n,n^2$

解出
$$
\begin{cases}
&A(n)=1\\
&B(n)=\frac{n(n+1)}{2}\\
&C(n)=n
\end{cases}
$$
故
$$
S_n=a+na+\frac{n(n+1)}{2}b
$$
\end{example}
\newp{2}

递归式可以转化为和式\\

对于形如
$$
a_nT_n=b_nT_{n-1}+c_n
$$
的递归式,可以设\textbf{求和因子(summation factor)}
$$
s_n=\frac{\prod\limits_{i=1}^{n}a_i}{\prod\limits_{i=1}^{n}b_i}\cdot\frac{b_1}{a_n}
$$
然后同时乘上\textbf{求和因子}即可得出
$$
T_n=\frac{1}{s_na_n}\left(s_1b_1T_0+\sum_{k=1}^{n}c_ks_k \right)
$$
注意:求和因子不能为0

\subsection{和式的处理 MANIPULATION OF SUMS}

\textbf{和式的变换}
$$
\begin{aligned}
&\sum_{k\in K}c a_k=c\sum_{k\in K}a_k\\
&\sum_{k\in K}(a_k+b_k)=\sum_{k \in K}a_k +\sum_{k \in K}b_k\\
&\sum_{k\in K} a_k =\sum_{p(k) \in K} a_{p(k)}\\
&\sum_{k\in K }a_k +\sum_{k\in K'} a_k=\sum_{k\in K\cap K'}a_k + \sum_{k\in K\cup K'} a_k
\end{aligned}
$$
其中对于$n\in K$,有且仅有一个整数满足$p(k)=n$\\

\subsection{扰动法 (perturbation method)}

对一个和式记其为$S_n$,将其第一项和最后一项分离出来,用两种方法改写$S_{n+1}$。\\

类似于\textbf{算两次}法

\begin{example}
  如求和式
  $$
  S_n=\sum_{0\le k \le n} k2^k
  $$
  有
  $$
  \begin{aligned}
  S_n+(n+1)2^{n+1}&=S_{n+1}\\
  &=\sum_{0 \le k \le n}(k+1)2^{k+1}\\
  &=2S_n+\sum_{0\le k \le n} 2^{k+1}
  \end{aligned}
  $$
  有
  $$
  \begin{aligned}
  S_n&=(n+1)2^{n+1}-\sum_{0 \le k \le n} 2^{k+1}\\
  &=(n+1)2^{n+1}-\frac{2(1-2^{n+1})}{1-2}\\
  &=(n-1)2^{n+1}+2
  \end{aligned}
  $$
\end{example}

\subsection{多重和式 MULTIPLE SUMS }

基本性质
$$
\sum_j\sum_ka_{j,k}[P(j,k)]=\sum_{P(j,k)}a_{j,k}=\sum_{k}\sum_ja_{j,k}[P(j,k)]
$$

$$
\sum_{j\in J}\sum_{k\in K(j)} a_{j,k} =\sum_{k\in K'}\sum_{j\in J'(k)} a_{j,k}
$$

$$
\sum_{j\in J}a_{f(j)}=\sum_{j\in J,k\in K} a_k[f(j)=k]=\sum_{k\in K} a_k \sum_{j\in J} [f(j)=k]
$$

其中$f: J\rightarrow K$

\begin{example}
  求
$$
S_n=\sum_{1\le j <k\le n}\frac{1}{k-j}
$$
有
$$
\begin{aligned}
S_n&=\sum_{1\le k \le n} \sum_{1\le j<k}\frac{1}{k-j}\\
&=\sum_{1\le k \le n} \sum_{0<j\le k-1}\frac{1}{j}\\
&=\sum_{1\le k \le n} H_{k-1}
\end{aligned}
$$
不太好做,(可以交换求和次序解 )

考虑直接把$k-j$当成一个整体。
$$
\begin{aligned}
S_n&=\sum_{1\le j<k+j\le n}\frac{1}{k}\\
&=\sum_{1\le k \le n}\sum_{1\le j\le n-k} \frac{1}{k}\\
&=\sum_{1\le k \le n} \frac{n-k}{k}\\
&=nH_{n}-n
\end{aligned}
$$
有
$$
\sum_{0\le k <n} H_k=nH_{n}-n
$$
\end{example}

\newp{2}
思考:\\

对含$k+f(j)$的二重和式,可以考虑用$k-f(j)$替换$k$,并先对$j$求和比较好。\\

几何观点: 按对角线求和。

\subsection{一般性的方法 GENERAL METHODS }

以
$$
S_n=\sum_{0\le k \le n} k^2
$$
为例

\subsubsection{归纳法}
如果注意到
$$
S_n=\frac{n(n+\frac{1}{2})(n+1)}{3}
$$
就可以使用数学归纳法

\subsubsection{扰动法}

观察
$$
\begin{aligned}
\sum_{0\le k \le n} k^2+(n+1)^2&=S_{n+1}\\
&=\sum_{1 \le  k\le n+1} k^2\\
&=\sum_{0 \le k \le n}(k+1)^2\\
&=\sum_{0\le k \le n} k^2+2\sum_{0\le k \le n} k+n+1
\end{aligned}
$$
虽然没有成功,但注意到我们,成功地解出了
$$
\sum_{0 \le k \le n} k
$$
考虑对
$$
T_n=\sum_{0 \le k \le n} k^3
$$
操作,有
$$
\begin{aligned}
T_n+(n+1)^3&=T_{n+1}\\
&=\sum_{1 \le k \le n+1} k^3\\
&=\sum_{0 \le k \le n} (k+1)^3\\
&=T_n+3S_n+3\sum_{0 \le k \le n}k+n+1
\end{aligned}
$$
得到
$$
\begin{aligned}
3S_n&=(n+1)^3-3\frac{n(n+1)}{2}-(n+1)\\
&=(n+1)\left(n^2+\frac{1}{2}n\right)\\
&=n(n+\frac{1}{2})(n+1)
\end{aligned}
$$

\subsubsection{成套方法}

有
$$
\begin{aligned}
&R_0=d\\
&R_n=R_{n-1}+an^2+bn+c
\end{aligned}
$$
其解的一般形式为
$$
R_n=aA(n)+bB(n)+cC(n)+dD(n)
$$
设$R_n=1,n,n^2,n^3$

解得
$$
\begin{cases}
A(n)=\frac{n(n+\frac{1}{2})(n+1)}{3}\\
B(n)=\frac{1}{2}(n^2+n)\\
C(n)=n\\
D(n)=1
\end{cases}
$$
故
$$
R_n=A(n)
$$
事实上对(a,b,c,d)赋值更简单。

\subsubsection{微积分法}

求
$$
\begin{aligned}
S_n-\int_0^nx^2 dx&=\sum_{1 \le k \le n}\left(k^2-\int_{k-1}^{k} x^2dx \right)\\
&=\sum_{1\le k \le n}(k-\frac{1}{3})\\
&=\frac{n(n+1)}{2}+\frac{n}{3}
\end{aligned}
$$

\subsubsection{展开和收缩}

转化为二重和式,以简化通项。
$$
\begin{aligned}
S_n&=\sum_{1 \le k \le n} k^2\\
&=\sum_{1 \le k \le n}k \sum_{1\le j \le k}1 \\
&=\sum_{1 \le j \le n} \sum_{j\le k \le n} k\\
&=\sum_{1 \le j \le n} \frac{(j+n)(n-j+1)}{2}\\
&=\frac{n^3+n^2}{2}+\frac{1}{2}\frac{n(n+1)}{2}- \frac{1}{2} S_n
\end{aligned}
$$

\subsubsection{有限微积分}

有$k^2=k^{\underline{2}}+k^{\underline{1}}$\\

故
$$
\begin{aligned}  
\sum_{0 \le k \le n} k^2 &= \sum_{0 \le k < n+1} k^{\underline{2}} + k^{\underline{1}} \\
&= \left. \left( \frac{k^{\underline{3}}}{3} + \frac{k^{\underline{2}}}{2} \right) \right|_{0}^{n+1} \\
&= \left( \frac{(n+1)^{\underline{3}}}{3} + \frac{(n+1)^{\underline{2}}}{2} \right) \\
&= \frac{(n+1)(n+\frac{1}{2})n}{3}.  
\end{aligned}
$$


\subsection{有限微积分}

类似微分算子$D$
$$
D f(x)=\lim\limits_{h\rightarrow 0}\frac{f(x+h)-f(x)}{h}
$$
定义差分算子$\Delta$
$$
\Delta f(x)= f(x+1)-f(x)
$$
定义\textbf{下降阶乘幂 (falling factorial power)}
$$
x^{\underline{m}}=x(x-1)\cdots(x-m+1)\qquad (m\ge 0 \in Z)
$$
和\textbf{上升阶乘幂 (rising factorial power)}
$$
x^{\overline{m}}=x(x+1)\cdots(x+m-1)\qquad (m\ge 0 \in Z)
$$
注意到:$n!=n^{\underline{n}}=1^{\overline{n}}$

有
$$
\Delta(x^{\underline{m}})=mx^{\underline{m-1}}
$$
类比积分,我们定义\textbf{不定和式 (indefinite sum)}
$$
\sum g(x)\delta x
$$
满足
$$
g(x)=\Delta f(x)\iff \sum g(x)\delta x=f(x)+C
$$
其中$C$为满足$p(x+1)=p(x)$的任意一个函数$p(x)$。\\

有限微积分有确定的和式 (sum)
$$
\sum_a^b g(x)\delta x=f(x)|_a^b=f(b)-f(a)
$$
有以下性质
$$
\begin{aligned}
&\sum_a^b g(x)\delta x= \sum_{a\le k < b} g(x) \qquad a\le b\\
&\sum_a^{b}g(x) \delta x =-\sum_{b}^ag(x)\delta x\\
&\sum_a^b+\sum_b^c=\sum_a^c
\end{aligned}
$$
\textbf{并且阶乘幂满足二项式定理}\\

负指数的下降阶乘幂定义如下
$$
x^{\underline{-m}}= \frac{1}{(x+1)(x+2)\cdots(x+m)} \qquad m>0
$$

从而有以下性质
$$
x^{\underline{m+n}}=x^{\underline{m}}(x-m)^{\underline{n}}
$$

$$
\sum_{a}^{b} x^{\underline{m}}  \delta x=\left. \frac{x^{\underline{m+1}}}{m+1}\right|_a^b, \qquad (m\neq -1)
$$

若$m=-1$ 则为$H_b-H_a$

% ![image-20250206212422837](Concrete-Mathematics.assets/image-20250206212422837.png)
\begin{tabular}{|c|c|c|c|}
  \hline
  \( f = \sum g \) & \( \Delta f = g \) & \( f = \sum g \) & \( \Delta f = g \) \\
  \hline
  \( x^{\underline{0}} = 1 \)    & 0                & \( 2^x \)        & \( 2^x \)          \\
  \hline
  \( x^{\underline{1}} = x \) & 1                & \( c^x \)        & \( (c - 1)c^x \)   \\
  \hline
  \( x^{\underline{2}} = x(x - 1) \) & \( 2x \)      & \( \frac{c^x}{c - 1} \) & \( c^x \)     \\
  \hline
  \( x^{\underline{m}} \)         & \( mx^{m-1} \)   & \( cu \)         & \( c\Delta u \)    \\
  \hline
  \( \frac{x^{\underline{m+1}}}{m + 1} \) & \( x^m \) & \( u + v \)      & \( \Delta u + \Delta v \) \\
  \hline
  \( H_x \)         & \( x^{\underline{-1}} = \frac{1}{x + 1} \) & \( uv \) & \( u\Delta v + Ev\Delta u \) \\
  \hline
  \end{tabular}

\subsubsection{分部求和 (summation by parts)}

有
$$
\Delta(u(x)v(x))=u(x)\Delta v(x)+Ev(x)\Delta u(x)
$$
其中,\textbf{E为移位算子 (shift operator)}$Ef(x)=f(x+1)$

简记为
$$
\Delta(uv)=u\Delta v+Ev \Delta u
$$
从而有
$$
\sum u\Delta v=uv-\sum Ev\Delta u
$$
如
\begin{example}
  $$
\begin{aligned}
\sum_{k=0}^{n}k 2^k&=\sum_{0}^{n+1}x2^x\delta x\\
&=\sum_{0}^{n+1}x\delta 2^x\\
&=(n+1)2^{n+1}-\sum_{0}^{n+1}2^{x+1}\delta x\\
&=(n+1)2^{n+1}-2^{n+2}+2\\
&=(n-1)2^{n+1}+2
\end{aligned}
$$
\end{example}

\begin{example}
  $$
\begin{aligned}
\sum_{0\le k< n} k H_k&=\sum_{0}^{n} xH_x \delta x\\
&=\frac{1}{2}\left( \sum_0^n H_x\delta x^{\underline{2}}  \right)\\
&=\frac{1}{2}\left( x^{\underline{2}} H_n- \sum_0^n x \delta x  \right)\\
&=\frac{1}{2}\left( x^{\underline{2}} H_n- \frac{n^{\underline{2}}}{2}  \right)\\
&=\frac{n^{\underline{2}}}{2} \left(H_n-\frac{1}{2} \right)
\end{aligned}
$$
\end{example}

\subsection{无限和式 INFINITE SUMS}

容易发现
$$
\sum_{k\ge0} x^k=
\begin{cases}
\frac{1}{1-x},& 0\le x<1\\
\infty ,& x\ge1
\end{cases}
$$
交错和
$$
\sum_{k\in K}a_k =\sum_{k\in K}a_k^{+}-\sum_{k\in K} a_k^{-}
$$
设$A^{+}=\sum_{k\in K}a_k^{+} $,类似定义$A^{-}$\\

a.若均有限的值,则称为绝对收敛。\\
b.若$A^{+}=\infty$,而后者为有限的值,则称发散于$+\infty$,反之发散于$-\infty$\\
c.否则不做定义。\\

只要我们处理的是刚才所定义的绝对收敛的和式,这一章里的所有操作都完全成立.\\

对复数分实部和虚部计算即可。\\

\chapter{概率论}
\section{基本概念和公式}
对概率运算规定一些简单的基本法则:\\
1. 设$A$是随机事件,则$0 \leq P(A) \leq 1$,\\
2. 设$\Omega$为必然事件,则$P(\Omega)=1$,\\
3. 若事件$A$和$B$不相容,则$P(A\cup B)=P(A)+P(B)$,\\
可推广至无穷:$$P(\bigcup_{i=1}^{n}A_i)=\sum_{i=1}^{\infty}P(A_i)$$.\\
4. 一般情况下,$P(A\cup B)=P(A)+P(B)-P(AB)$,$P(A\cup B \cup C)=P(A)+P(B)+P(C)-P(AB)-P(AC)-P(BC)+P(ABC)$ \\
5. $P(\overline{A})=1-P(A)$\\
6. $P(A-B)=P(A)-P(AB)$\\

\begin{theorem}全概率公式\\
  设$B_1,B_2,...B_n$是样本空间$\Omega$中的\textbf{两两不相容}的一组事件,
  即$B_i B_j=\phi$,$i\neq j$,且满足$\bigcup_{i=1}^{n}B_i=\Omega$,则称$B_1,B_2,...B_n$
  是样本空间$\Omega$的一个分割(又称为完备事件群,英文为partition)。
  设$\{B_1,B_2,...B_n\}$是样本空间$\Omega$的一个分割,$A$为$\Omega$的一个事件,则
  $$
  P(A)=\sum_{i=1}^{n}P(A|B_i)P(B_i)
  $$
\end{theorem}

\begin{theorem}贝叶斯公式\\
  设$\{B_1, B_2, ...B_n\}$是样本空间的一个分割,$A$为$\Omega$中的一个事件,$P(B_i)>0$,$i=1,2,...,n$,$P(A)>0$,则
$$
P(B_i|A)=\frac{P(A|B_i)P(B_i)}{\sum_{j=1}^{n}P(A|B_j)P(B_j)}
$$
用于因果转换.
\end{theorem}

\begin{definition}事件的独立性\\
  设$A$,$B$是随机试验中的两个事件,若满足$P(AB)=P(A)P(B)$,则称事件$A$和$B$相互独立。\\
  判断事件的独立,应该是\textbf{从实际出发},如果能够判断事件$B$的发生与否对事件$A$的发生与否不产生影响,则事件$A$,$B$即为独立。\\
  设$\widetilde{A}$表示事件$A$发生和不发生之一,$\widetilde{B}$表示事件$B$发生和不发生之一。有独立性的定义可推至$P(\widetilde{A}\widetilde{B})=P(\widetilde{A})P(\widetilde{B})$(一共有四个等式)。可推广至:
$$
P(\widetilde{A}_1\widetilde{A}_2...\widetilde{A}_n)=P(\widetilde{A}_1)...P(\widetilde{A}_n)
$$
上面有$2^n$个等式。\\
独立一定相容
\end{definition}

\textbf{重要公式与结论}
$$
\begin{aligned}
&(1)\ P(\overline{A})=1-P(A)\\
\\
&(2)\ P(A \cup B)=P(A)+P(B)-P(AB)\\
\\
&(3)\ P(A\cup B \cup C)=P(A)+P(B)+P(C)-P(AB)-P(AC)-P(BC)+P(ABC)\\
\\
&(4)\ P(A-B)=P(A)-P(AB)\\
\\
&(5)\ P(A\overline{B})=P(A)-P(AB),P(A)=P(AB)+P(A\overline{B}),\\
&\ \ \ \ \ \ P(A\cup B)=P(A)+P(\overline{A}B)=P(AB)+P(A\overline{B})+P(\overline{A}B)\\
\\
&(6)\ P(\overline{A}_1|B)=1-P(A_1|B),P(A_1\cup A_2|B)=P(A_1|B)+P(A_2|B)-P(A_1A_2|B)\\
&\ \ \ \ \ P(A_1A_2|B)=P(A_1|B)P(A_2|A_1B)\\
\\
&(7)\ 若A_1,A_2,...A_n相独立,则P(\bigcap_{i=1}^{n}A_i)=\prod_{i=1}^{n}P(A_i),P(\bigcup_{i=1}^{n}A_i)=\prod_{i=1}^{n}(1-P(A_i))
\end{aligned}
$$

随机变量(Random variable):值随机会而定的变量,研究随机试验的一串事件。可按维数分为一维、二维至多维随机变量。按性质可分为离散型随机变量以及连续型随机变量。\\
分布(Distribution):事件之间的联系,用来计算概率。\\
示性函数(Indication function):$I_A(\omega)=\begin{cases}
    1& \omega \in A \\ 
    0& \text{ 反之}
   \end{cases}$,事件$A$有随机变量$I_A$表示出来,$I_A$称为事件$A$的示性函数。\\

\begin{definition}
  概率函数:\\
  设$X$为一随机变量,其全部可能值为$\{a_1, a_2,...\}$,则$p_i=P(X=a_i),i=1,2,...$称为$X$的概率函数。
\end{definition}

\begin{definition}
  概率分布函数:\\
  定义:设$X$为一随机变量,则函数
$$
F(X)=P(X\leq x)\quad(-\infty<x<\infty)
$$
称为$X$的分布函数。(注:这里并未限定$X$为离散型的,它对任何随机变量都有定义。)\\

\end{definition}
性质:\\
   $F(x)\text{是单调非降的:当}$ $x_1<x_2$时,有$F(x_1)\leq F(X_2)$.\\
   当$x \rightarrow \infty$时,$F(x)\rightarrow1$;当$x \rightarrow-\infty$时,$F(x)\rightarrow0$.\\
   离散型随机变量分布函数:\\
   对于离散型随机变量,$F(X)=P(X\leq x)=\sum_{\{i|a_i\leq x\}}p_i, \quad p_i=P(X=i)=F(i)-F(i-1)$。\\
   1. 连续型随机变量:设$X$为一随机变量,如果$X$不仅有无限个而且有不可数个值,则称$X$为一个连续型随机变量。

   \begin{definition}
    概率密度函数:\\
    设连续型随机变量$X$有概率分布函数$F(x)$,则$F(x)$的导数$f(x)=F'(x)$称为$X$的概率密度函数。
   \end{definition}
   性质\\
     1.对于所有的$-\infty<x<+\infty$,有$f(x)\ge 0$;\\
     2.$\int_{-\infty}^{+\infty}f(x)dx=1$;\\
     3.对于任意的$-\infty<a\leq b<+\infty$,有$P(a\leq X\leq b)=F(b)-F(a)=\int_{a}^{b}f(x)dx$.\\
   注:\\
     1.对于任意的$-\infty<x<+\infty$,有$P(X=x)=\int_{x}^{x}f(u)du=0$.\\
     2.假设有总共一个单位的质量连续地分布在$a\leq x\leq b$上,那么$f(x)$表示在点$x$的质量密度且$\int_{c}^{d}f(x)dx$表示在区间$[c, d]$上的全部质量。\\
     \begin{definition}
      概率分布函数:\\
      设$X$为一连续型随机变量,则
      $$
      F(x)=\int_{-\infty}^xf(u)du,\quad-\infty<x<+\infty
      $$
     \end{definition}

\section{重要公式与结论}
二项分布$X\sim B(n,p)$的期望为np,方差为np(1-p)\\
均匀分布$X\sim U(a,b)$的期望为$\frac{a+b}{2}$ ,方差为$\frac{1}{12}(b-a)^2$\\
\begin{definition}
  边缘分布:\\
  因为$X$的每个分量$X_i$都是一维随机变量,故它们都有各自的分布$F_i\ (i=1,...,n)$,这些都是一维分布,称为随机向量$X$或其分布$F$的边缘分布。\\
  离散随机变量:\\
  $$
  \begin{aligned}
    p_X(x_i)&=P(X=x_i)\\
    &=\sum_{j}^{m}P(X=x_i,Y=y_j)\\
    &=\sum_{j}^{m}p_{ij}=p_{i\cdot},\quad i=1,2,...,n
  \end{aligned}
  $$

  $$
  \begin{aligned}
    p_Y(y_i)&=P(Y=y_i)\\
    &=\sum_{i}^{m}P(X=x_i,Y=y_j)\\
    &=\sum_{i}^{m}p_{ij}=p_{j\cdot},\quad j=1,2,...,n
  \end{aligned}
  $$
  连续随机变量:\\
  为求某分量$X_i$的概率密度函数,只需把$f(x_1,...,x_n)$中的$x_i$固定,然后对$x_1,...,x_{i-1},x_{i+1},...,x_n$在$-\infty$到$\infty$之间做定积分,如
  $$
  (X,Y)\sim f(x, y)\\
  f_X(u)=\int^{+\infty}_{-\infty}f(u,v)dv\\
  f_Y(u)=\int^{+\infty}_{-\infty}f(u,v)du\\
  $$
\end{definition}

\begin{definition}
  离散型随机变量的条件分布:设$(X,Y)$为二维离散型随机变量,对于给定的事件$\{Y=y_j\}$,其概率$P(Y=y_j)>0$,则称\\
   $$
   P(X=x_i|Y=y_j)=\frac{P(X=x_i,Y=y_j)}{P(Y=y_j)}=\frac{p_{ij}}{p_{\cdot j}},\quad i=1,2,...
   $$
   为在给定$Y=y_j$的条件下$X$的条件分布律。类似的,称
   $$
   P(Y=y_i|X=x_j)=\frac{P(X=x_i,Y=y_j)}{P(X=x_j)}=\frac{p_{ij}}{p_{i\cdot}},\quad j=1,2,...
   $$
   为在给定$X=x_j$的条件下$Y$的条件分布律。

 连续型随机变量的条件分布:设$(X,Y)$为二维连续型随机变量,对于给定条件$Y=y$下的条件概率密度为\\
   $$
   f_{X|Y}(x|y)=\frac{f(x,y)}{f_Y(y)}, \quad f_Y(y)>0.\\
   $$
   类似的,在$X=x$下的条件概率密度为
   $$
   f_{Y|X}(y|x)=\frac{f(x,y)}{f_X(x)}, \quad f_X(x)>0.\\
   $$
\end{definition}
可推广\\


\begin{definition}
  随机变量的独立性\\
  称随机变量$X_1, ...,X_n$相互独立,\\
  1.离散型随机变量\\
     则联合分布律等于各自的边缘分布律的乘积,即
     $$
     P(X_1=x_1,...,X_n=x_n)=P(X_1=x_1)...P(X_n=x_n)
     $$
     其中$(x_1,...x_n)$为$(X_1,...,X_n)$的值域中的任意一点。
  2.连续型随机变量\\
     则联合密度等于各自的边缘密度的乘积,即
     $$
     f(x_1,...,x_n)=f_1(x_1)...f_n(x_n),\quad \forall(x_1,...,x_n)\in R ^n
     $$
   3.一般地\\
     设$X_1,...,X_n$为$n$个随机变量,如果它们的联合分布函数等于各自边缘分布函数的乘积,即
     $$
     F(X_1, ...,x_n)=F_1(x_1)...F_n(x_n),\quad \forall (x_1,...,x_n)\in R^n
     $$
     则称随机变量$X_1, ...,X_n$相互独立。
\end{definition}

\textbf{以下内容才是重点!!!!!}\\
\textbf{以下内容才是重点!!!!!}\\
\textbf{以下内容才是重点!!!!!}\\

\subsection{数学期望(均值)与方差}
\begin{definition}
  数学期望\\
  设随机变量$X$只取有限个可能值$a_1,...,a_m$,其概率分布为$P(X=a_i)=p_i\ (i=1,...,m)$.\\
  则$X$的数学期望记作$EX$或$E(X)$, 定义为$E(X)=a_1p_1+a_2p_2+...+a_mp_m$.\\
  数学期望也常称为均值,即指以概率为权的加权平均。\\
  1.离散型变量的数学期望:$E(X)=\sum^\infty_{i=1}a_ip_i.$(当级数绝对收敛,即$\sum_{i=1}^\infty|a_i|p_i<\infty$)\\
  2.连续型变量的数学期望:$E(X)=\int_{-\infty}^\infty xf(x)dx$.(当$\int_{-\infty}^\infty |x|f(x)dx<\infty$)
\end{definition}
性质\\
  1.若干个随机变量之和的期望等于各变量的期望值和,即
     $$
     E(X_1+X_2+...+X_n)=E(X_1)+E(X_2)+...+E(X_n).
     $$
  2.若干个独立随机变量之积的期望等于各变量的期望之积,即
     $$
     E(X_1X_2...X_n)=E(X_1)E(X_2)...E(X_n).
     $$
  3.设随机变量$X$为离散型,有分布$P(X=a_i)=p_i(i=1,2,...)$;或者为连续型,有概率密度函数$f(x)$.则
    $$
      E(g(x))=\sum_ig(a_i)p_i\quad (\text{当}\sum_i|g(a_i)|p_i<\infty \text{时})
    $$
      或
    $$
      E(g(x))=\int_{-\infty}^\infty g(x)f(x)dx \quad (\text{当}\int_{-\infty}^{\infty}|g(x)|f(x)dx<\infty\text{时})
   $$
  4.若$c$为常数,则$E(cX)=cE(X)$.\\
  \begin{definition}
    条件数学期望\\
    随机变量Y的条件期望就是它在给定的某种附加条件下的数学期望。\\
    $E(Y|x)=\int_{-\infty}^{\infty}yf(y|x)dy$.它反映了随着$X$取值$x$的变化$Y$的平均变化的情况如何。\\
    在统计上,常把条件期望$E(Y|x)$作为$x$的函数,称为$Y$对$X$的回归函数。
  \end{definition}
  性质:
    1.$E(Y)=\int_{-\infty}^{\infty}E(Y|x)f_X(x)dx$.\\
    2. $E(Y)=E[E(Y|X)]$.

\begin{definition}
  方差与标准差\\
  设$X$为随机变量,分布为$F$,则$Var(X)=E(X-EX)^2$称为$X$(或分布$F$)的方差,\\
  其平方根$\sqrt{Var(X)}$(取正值)称为$X$(或分布$F$)的标准差。
\end{definition}
性质:
  1.$Var(X)=E(X^2)-(EX)^2$.\\
  2. 常数的方差为0,即$Var(c)=0$.\\
  3.若$c$为常数,则$Var(X+c)=Var(X)$.\\
  4.若$c$为常数,则$Var(cX)=c^2Var(X)$.\\
  5.独立随机变量和的方差等于各变量方差和,即$Var(X_1+...+X_n)=Var(X_1)+...+Var(X_n)$.\\





\section{期望经典问题入门}
\url{https://notes.sshwy.name/Math/Expectation/Classic/#E-%E7%BB%8F%E5%85%B8%E9%A2%98}
重要公式与结论
1. 期望具有线性性\\
2. 独立事件的期望有
$$
E(XY)=EXEY
$$
3.
$$
E(X)=E(E(X|Y))
$$
\subsection{普通}
\begin{proposition}
  有 $n$ 个随机变量$\langle X_n \rangle$,每个随机变量量都是从[1,m]中随机一个整数,$\max{\langle X_n \rangle}$的期望
  为
  $$
  m-\frac{1}{m^n}\sum_{i=1}^{m-1}i^{n}
  $$。
\end{proposition}

\begin{proof}
  设$Y=\max{\langle X_n \rangle}$ , 有
  $$
  \begin{aligned}
    EY&=\sum_{1\le i \le m} P(Y=i)i\\
    &=\sum_{1\le i\le m} i(F_y(i)-F_y(i-1))\\
    &=\frac{1}{m^n}\left( \sum_{1\le i\le m} i^{n+1}-i\cdot (i-1)^{n}\right) \\
    &=\frac{1}{m^n}\left( \sum_{1\le i\le m} i^{n+1}-\sum_{0\le i\le m-1} i^{n+1} +i^{n}\right)\\
    &=\frac{1}{m^n}\left( m^{n+1}-\sum_{0\le i\le m-1}i^{n}\right)\\
    &=m-\frac{1}{m^n}\sum_{i=1}^{m-1}i^{n}
  \end{aligned}
  $$
\end{proof}

\begin{proposition}
  概率为$p$的事件期望$\frac{1}{p}$次发生.
\end{proposition}
\begin{proof}
  设随机变量$X$表示其在第$x$次发生. 
  $$
  \begin{aligned}
    EX&=\sum_{i=1}^{\infty} P(X=i)i\\
      &=\sum_{i=1}^{\infty} P(X=i) \sum_{j=1}^{i} 1\\
      &=\sum_{j=1}^{\infty} P(X\ge j)\\
      &=\sum_{j=1}^{\infty} (1-p)^{j-1}\\
      &=\sum_{j=0}^{\infty} (1-p)^j\\
      &=\frac{1}{1-(1-p)}\\
      &=\frac{1}{p}
  \end{aligned}
  $$
\end{proof}

\begin{proposition}
  现在红包发了一个$w$元的红包,有 $n$ 个人来抢(均匀分布)。那么请问第 $k$ 个人期望抢到$\frac{w}{2^k}$.
\end{proposition}

\begin{proof}
设第$k$个人抢到$X$,前面的人抢$Y$.
有
$$
E(X)=E(E(X|Y))=E(\frac{w-Y}{2})=\frac{w}{2}-\frac{1}{2}E(Y)
$$
然后容易解出答案.\\
离散情况这样难以求解,在取球一节有其他解法\\
\end{proof}

\begin{proposition}赠券收集问题\\
  一个$n$面的骰子,期望$nH_n$ 次能使得每一面都被掷到。
\end{proposition}

\begin{proof}
$t_x$为设已经出现了$x-1$面,掷出第$x$面的次数.\\
设$T=\sum t$.有
$$
E(T)=\sum_{i=1}^{n}E(t_i)
$$
由于掷出第$x$面的概率为$\frac{n-x+1}{n}$,固
$$
\begin{aligned}
  E(T)&=\sum_{i=1}^{n}E(t_i)\\
  &=\sum_{x=1}^{n} \frac{n-x+1}{n}\\
  &=n\cdot H_n
\end{aligned}
$$
\end{proof}
同时他们也是相互独立的.\\


\subsection{拿球}
\begin{proposition}
  箱子里有$n$个球$1,2,⋯,n$, 你要从里面拿$m$次球,拿了后不放回,取出的数字之和的期望为$\frac{m(n+1)}{2}$。
\end{proposition}

\begin{proof}
  设随机变量$x_i$:\\
  $$x_i=\begin{cases}
  i\quad&, \text{if i is chosen}\\
  0\quad&, \text{if i isn't chosen}
  \end{cases}
  $$
  那么有
  $$
  \begin{aligned}
    E\left(\sum_{i=1}^{n}x_i\right)&=\sum_{i=1}^{n}E(x_i)\\
    &=\sum_{i=1}^{n} \frac{m}{n} i\\
    &=\frac{m(n+1)}{2}
  \end{aligned}
  $$
  发现是否放回不影响期望\\
\end{proof}


\begin{proposition}
  箱子里有$n$个球$1,2,⋯,n$,你要从里面拿$m$次球,拿了后以$p_1$的概率放回,$p_2$
  的概率放回两个和这个相同的球(相当于增加一个球),取出的数字之和的期望为 $\frac{m(n+1)}{2}$。
\end{proposition}

\begin{proof}
  设$x_i$为第$i$个球的贡献, $y_i$ 为其被拿出来的次数 ,那么$x_i=i\cdot y_i$
  $$
  E\left(\sum_{i=1}^{n}x_i\right)=\sum_{i=1}^{n}E(y_i)\cdot i
  $$
  因为$E(y_i)=E(y_j)$, $\sum y_i =m$ 得出$E(y_i)=\frac{m}{n}$
  故上式答案为$\frac{m(n+1)}{2}$
\end{proof}

\subsection{游走}
\begin{proposition}
  在一条$n$个点的链上游走,求从一端走到另一端的期望步数。
\end{proposition}

\begin{proof}
  
\end{proof}

\subsection{解题方法}
1.贡献法,\\
若不行可以尝试更换计算贡献的东西(如边->点)\\

\chapter{杂}
\section{最小二乘法}
误差的平方和最小\\
$$
  L=\sum_{i=1}^{n} (y_i-f(x_i))^2
$$
拟合直线$y=kx+b$ 
$$
  \begin{cases}
    k&=\frac{\sum xy- n \overline{x}\overline{y}}{\sum x^2-n\overline{x}^2}\\
    b&=\overline{y}-k\overline{x}
  \end{cases}
$$

\end{document}

\section{常用幂级数}
$$
\begin{aligned}
e^x&=\sum_{n=0}^{\infty}\frac{x^n}{n!} \qquad& x\in(-\infty,+\infty)\\
\sin x&=\sum_{n=0}^{\infty}(-1)^n\frac{x^{2n+1}}{(2n+1)!} \qquad& x\in(-\infty,+\infty)\\
\cos x&=\sum_{n=0}^{\infty}(-1)^n\frac{x^{2n}}{(2n)!} \qquad& x\in(-\infty+-\infty)\\
\frac{1}{x+1}&=\sum_{n=0}^{\infty}(-1)^nx^n \qquad& x\in(-1,1)\\
\ln{(x+1)}&=\sum_{n=0}^{\infty}(-1)^n\frac{x^{n+1}}{n+1}\qquad&x\in(-1,1]
\end{aligned}
$$