\section{基本概念和公式}
对概率运算规定一些简单的基本法则:\\
1. 设$A$是随机事件,则$0 \leq P(A) \leq 1$,\\
2. 设$\Omega$为必然事件,则$P(\Omega)=1$,\\
3. 若事件$A$和$B$不相容,则$P(A\cup B)=P(A)+P(B)$,\\
可推广至无穷:$$P(\bigcup_{i=1}^{n}A_i)=\sum_{i=1}^{\infty}P(A_i)$$.\\
4. 一般情况下,$P(A\cup B)=P(A)+P(B)-P(AB)$,$P(A\cup B \cup C)=P(A)+P(B)+P(C)-P(AB)-P(AC)-P(BC)+P(ABC)$ \\
5. $P(\overline{A})=1-P(A)$\\
6. $P(A-B)=P(A)-P(AB)$\\

\begin{theorem}全概率公式\\
  设$B_1,B_2,...B_n$是样本空间$\Omega$中的\textbf{两两不相容}的一组事件,
  即$B_i B_j=\phi$,$i\neq j$,且满足$\bigcup_{i=1}^{n}B_i=\Omega$,则称$B_1,B_2,...B_n$
  是样本空间$\Omega$的一个分割(又称为完备事件群,英文为partition)。
  设$\{B_1,B_2,...B_n\}$是样本空间$\Omega$的一个分割,$A$为$\Omega$的一个事件,则
  $$
  P(A)=\sum_{i=1}^{n}P(A|B_i)P(B_i)
  $$
\end{theorem}

\begin{theorem}贝叶斯公式\\
  设$\{B_1, B_2, ...B_n\}$是样本空间的一个分割,$A$为$\Omega$中的一个事件,$P(B_i)>0$,$i=1,2,...,n$,$P(A)>0$,则
$$
P(B_i|A)=\frac{P(A|B_i)P(B_i)}{\sum_{j=1}^{n}P(A|B_j)P(B_j)}
$$
用于因果转换.
\end{theorem}

\begin{definition}事件的独立性\\
  设$A$,$B$是随机试验中的两个事件,若满足$P(AB)=P(A)P(B)$,则称事件$A$和$B$相互独立。\\
  判断事件的独立,应该是\textbf{从实际出发},如果能够判断事件$B$的发生与否对事件$A$的发生与否不产生影响,则事件$A$,$B$即为独立。\\
  设$\widetilde{A}$表示事件$A$发生和不发生之一,$\widetilde{B}$表示事件$B$发生和不发生之一。有独立性的定义可推至$P(\widetilde{A}\widetilde{B})=P(\widetilde{A})P(\widetilde{B})$(一共有四个等式)。可推广至:
$$
P(\widetilde{A}_1\widetilde{A}_2...\widetilde{A}_n)=P(\widetilde{A}_1)...P(\widetilde{A}_n)
$$
上面有$2^n$个等式。\\
独立一定相容
\end{definition}

\textbf{重要公式与结论}
$$
\begin{aligned}
&(1)\ P(\overline{A})=1-P(A)\\
\\
&(2)\ P(A \cup B)=P(A)+P(B)-P(AB)\\
\\
&(3)\ P(A\cup B \cup C)=P(A)+P(B)+P(C)-P(AB)-P(AC)-P(BC)+P(ABC)\\
\\
&(4)\ P(A-B)=P(A)-P(AB)\\
\\
&(5)\ P(A\overline{B})=P(A)-P(AB),P(A)=P(AB)+P(A\overline{B}),\\
&\ \ \ \ \ \ P(A\cup B)=P(A)+P(\overline{A}B)=P(AB)+P(A\overline{B})+P(\overline{A}B)\\
\\
&(6)\ P(\overline{A}_1|B)=1-P(A_1|B),P(A_1\cup A_2|B)=P(A_1|B)+P(A_2|B)-P(A_1A_2|B)\\
&\ \ \ \ \ P(A_1A_2|B)=P(A_1|B)P(A_2|A_1B)\\
\\
&(7)\ 若A_1,A_2,...A_n相独立,则P(\bigcap_{i=1}^{n}A_i)=\prod_{i=1}^{n}P(A_i),P(\bigcup_{i=1}^{n}A_i)=\prod_{i=1}^{n}(1-P(A_i))
\end{aligned}
$$

随机变量(Random variable):值随机会而定的变量,研究随机试验的一串事件。可按维数分为一维、二维至多维随机变量。按性质可分为离散型随机变量以及连续型随机变量。\\
分布(Distribution):事件之间的联系,用来计算概率。\\
示性函数(Indication function):$I_A(\omega)=\begin{cases}
    1& \omega \in A \\ 
    0& \text{ 反之}
   \end{cases}$,事件$A$有随机变量$I_A$表示出来,$I_A$称为事件$A$的示性函数。\\

\begin{definition}
  概率函数:\\
  设$X$为一随机变量,其全部可能值为$\{a_1, a_2,...\}$,则$p_i=P(X=a_i),i=1,2,...$称为$X$的概率函数。
\end{definition}

\begin{definition}
  概率分布函数:\\
  定义:设$X$为一随机变量,则函数
$$
F(X)=P(X\leq x)\quad(-\infty<x<\infty)
$$
称为$X$的分布函数。(注:这里并未限定$X$为离散型的,它对任何随机变量都有定义。)\\

\end{definition}
性质:\\
   $F(x)\text{是单调非降的:当}$ $x_1<x_2$时,有$F(x_1)\leq F(X_2)$.\\
   当$x \rightarrow \infty$时,$F(x)\rightarrow1$;当$x \rightarrow-\infty$时,$F(x)\rightarrow0$.\\
   离散型随机变量分布函数:\\
   对于离散型随机变量,$F(X)=P(X\leq x)=\sum_{\{i|a_i\leq x\}}p_i, \quad p_i=P(X=i)=F(i)-F(i-1)$。\\
   1. 连续型随机变量:设$X$为一随机变量,如果$X$不仅有无限个而且有不可数个值,则称$X$为一个连续型随机变量。

   \begin{definition}
    概率密度函数:\\
    设连续型随机变量$X$有概率分布函数$F(x)$,则$F(x)$的导数$f(x)=F'(x)$称为$X$的概率密度函数。
   \end{definition}
   性质\\
     1.对于所有的$-\infty<x<+\infty$,有$f(x)\ge 0$;\\
     2.$\int_{-\infty}^{+\infty}f(x)dx=1$;\\
     3.对于任意的$-\infty<a\leq b<+\infty$,有$P(a\leq X\leq b)=F(b)-F(a)=\int_{a}^{b}f(x)dx$.\\
   注:\\
     1.对于任意的$-\infty<x<+\infty$,有$P(X=x)=\int_{x}^{x}f(u)du=0$.\\
     2.假设有总共一个单位的质量连续地分布在$a\leq x\leq b$上,那么$f(x)$表示在点$x$的质量密度且$\int_{c}^{d}f(x)dx$表示在区间$[c, d]$上的全部质量。\\
     \begin{definition}
      概率分布函数:\\
      设$X$为一连续型随机变量,则
      $$
      F(x)=\int_{-\infty}^xf(u)du,\quad-\infty<x<+\infty
      $$
     \end{definition}

\section{重要公式与结论}
二项分布$X\sim B(n,p)$的期望为np,方差为np(1-p)\\
均匀分布$X\sim U(a,b)$的期望为$\frac{a+b}{2}$ ,方差为$\frac{1}{12}(b-a)^2$\\
\begin{definition}
  边缘分布:\\
  因为$X$的每个分量$X_i$都是一维随机变量,故它们都有各自的分布$F_i\ (i=1,...,n)$,这些都是一维分布,称为随机向量$X$或其分布$F$的边缘分布。\\
  离散随机变量:\\
  $$
  \begin{aligned}
    p_X(x_i)&=P(X=x_i)\\
    &=\sum_{j}^{m}P(X=x_i,Y=y_j)\\
    &=\sum_{j}^{m}p_{ij}=p_{i\cdot},\quad i=1,2,...,n
  \end{aligned}
  $$

  $$
  \begin{aligned}
    p_Y(y_i)&=P(Y=y_i)\\
    &=\sum_{i}^{m}P(X=x_i,Y=y_j)\\
    &=\sum_{i}^{m}p_{ij}=p_{j\cdot},\quad j=1,2,...,n
  \end{aligned}
  $$
  连续随机变量:\\
  为求某分量$X_i$的概率密度函数,只需把$f(x_1,...,x_n)$中的$x_i$固定,然后对$x_1,...,x_{i-1},x_{i+1},...,x_n$在$-\infty$到$\infty$之间做定积分,如
  $$
  (X,Y)\sim f(x, y)\\
  f_X(u)=\int^{+\infty}_{-\infty}f(u,v)dv\\
  f_Y(u)=\int^{+\infty}_{-\infty}f(u,v)du\\
  $$
\end{definition}

\begin{definition}
  离散型随机变量的条件分布:设$(X,Y)$为二维离散型随机变量,对于给定的事件$\{Y=y_j\}$,其概率$P(Y=y_j)>0$,则称\\
   $$
   P(X=x_i|Y=y_j)=\frac{P(X=x_i,Y=y_j)}{P(Y=y_j)}=\frac{p_{ij}}{p_{\cdot j}},\quad i=1,2,...
   $$
   为在给定$Y=y_j$的条件下$X$的条件分布律。类似的,称
   $$
   P(Y=y_i|X=x_j)=\frac{P(X=x_i,Y=y_j)}{P(X=x_j)}=\frac{p_{ij}}{p_{i\cdot}},\quad j=1,2,...
   $$
   为在给定$X=x_j$的条件下$Y$的条件分布律。

 连续型随机变量的条件分布:设$(X,Y)$为二维连续型随机变量,对于给定条件$Y=y$下的条件概率密度为\\
   $$
   f_{X|Y}(x|y)=\frac{f(x,y)}{f_Y(y)}, \quad f_Y(y)>0.\\
   $$
   类似的,在$X=x$下的条件概率密度为
   $$
   f_{Y|X}(y|x)=\frac{f(x,y)}{f_X(x)}, \quad f_X(x)>0.\\
   $$
\end{definition}
可推广\\


\begin{definition}
  随机变量的独立性\\
  称随机变量$X_1, ...,X_n$相互独立,\\
  1.离散型随机变量\\
     则联合分布律等于各自的边缘分布律的乘积,即
     $$
     P(X_1=x_1,...,X_n=x_n)=P(X_1=x_1)...P(X_n=x_n)
     $$
     其中$(x_1,...x_n)$为$(X_1,...,X_n)$的值域中的任意一点。
  2.连续型随机变量\\
     则联合密度等于各自的边缘密度的乘积,即
     $$
     f(x_1,...,x_n)=f_1(x_1)...f_n(x_n),\quad \forall(x_1,...,x_n)\in R ^n
     $$
   3.一般地\\
     设$X_1,...,X_n$为$n$个随机变量,如果它们的联合分布函数等于各自边缘分布函数的乘积,即
     $$
     F(X_1, ...,x_n)=F_1(x_1)...F_n(x_n),\quad \forall (x_1,...,x_n)\in R^n
     $$
     则称随机变量$X_1, ...,X_n$相互独立。
\end{definition}

\textbf{以下内容才是重点!!!!!}\\
\textbf{以下内容才是重点!!!!!}\\
\textbf{以下内容才是重点!!!!!}\\

\subsection{数学期望(均值)与方差}
\begin{definition}
  数学期望\\
  设随机变量$X$只取有限个可能值$a_1,...,a_m$,其概率分布为$P(X=a_i)=p_i\ (i=1,...,m)$.\\
  则$X$的数学期望记作$EX$或$E(X)$, 定义为$E(X)=a_1p_1+a_2p_2+...+a_mp_m$.\\
  数学期望也常称为均值,即指以概率为权的加权平均。\\
  1.离散型变量的数学期望:$E(X)=\sum^\infty_{i=1}a_ip_i.$(当级数绝对收敛,即$\sum_{i=1}^\infty|a_i|p_i<\infty$)\\
  2.连续型变量的数学期望:$E(X)=\int_{-\infty}^\infty xf(x)dx$.(当$\int_{-\infty}^\infty |x|f(x)dx<\infty$)
\end{definition}
性质\\
  1.若干个随机变量之和的期望等于各变量的期望值和,即
     $$
     E(X_1+X_2+...+X_n)=E(X_1)+E(X_2)+...+E(X_n).
     $$
  2.若干个独立随机变量之积的期望等于各变量的期望之积,即
     $$
     E(X_1X_2...X_n)=E(X_1)E(X_2)...E(X_n).
     $$
  3.设随机变量$X$为离散型,有分布$P(X=a_i)=p_i(i=1,2,...)$;或者为连续型,有概率密度函数$f(x)$.则
    $$
      E(g(x))=\sum_ig(a_i)p_i\quad (\text{当}\sum_i|g(a_i)|p_i<\infty \text{时})
    $$
      或
    $$
      E(g(x))=\int_{-\infty}^\infty g(x)f(x)dx \quad (\text{当}\int_{-\infty}^{\infty}|g(x)|f(x)dx<\infty\text{时})
   $$
  4.若$c$为常数,则$E(cX)=cE(X)$.\\
  \begin{definition}
    条件数学期望\\
    随机变量Y的条件期望就是它在给定的某种附加条件下的数学期望。\\
    $E(Y|x)=\int_{-\infty}^{\infty}yf(y|x)dy$.它反映了随着$X$取值$x$的变化$Y$的平均变化的情况如何。\\
    在统计上,常把条件期望$E(Y|x)$作为$x$的函数,称为$Y$对$X$的回归函数。
  \end{definition}
  性质:
    1.$E(Y)=\int_{-\infty}^{\infty}E(Y|x)f_X(x)dx$.\\
    2. $E(Y)=E[E(Y|X)]$.

\begin{definition}
  方差与标准差\\
  设$X$为随机变量,分布为$F$,则$Var(X)=E(X-EX)^2$称为$X$(或分布$F$)的方差,\\
  其平方根$\sqrt{Var(X)}$(取正值)称为$X$(或分布$F$)的标准差。
\end{definition}
性质:
  1.$Var(X)=E(X^2)-(EX)^2$.\\
  2. 常数的方差为0,即$Var(c)=0$.\\
  3.若$c$为常数,则$Var(X+c)=Var(X)$.\\
  4.若$c$为常数,则$Var(cX)=c^2Var(X)$.\\
  5.独立随机变量和的方差等于各变量方差和,即$Var(X_1+...+X_n)=Var(X_1)+...+Var(X_n)$.\\




