\subsubsection{格}
\begin{definition}
  设R为非空集合A上的关系,如果R是自反的,反对称的,可传递的,则称R为A上的偏序关系,简称偏序,记作≤ 。
\end{definition}

\begin{definition}[覆盖 (Covering relation)]
  设 $(P,\le)$ 为偏序集,记严格顺序 $<$ 为 $x<y\iff x\le y\land x\neq y$。若对 $x,y\in P$ 有 $x<y$,
  且不存在 $z\in P$ 使得 $x<z<y$,
  则称 $y$ 覆盖(covers) $x$,记作 $x\lessdot y$。  
\end{definition}

\begin{definition}[哈斯图 (Hasse diagram)]
    哈斯图是有限偏序集 $(P,\le)$ 的传递约简 (transitive reduction) 的图形表示:
    将每个元素画为点;若 $y$ 覆盖 $x$,则在 $x$ 下方画一条无向(或向上)的线段连接 $x$ 与 $y$;
    省略所有反身和传递边。  
\end{definition}

\begin{definition}[最小元 (Least element)]
 若 $m\in P$ 满足对一切 $x\in P$ 均有 $m\le x$,则称 $m$ 为偏序集的\emph{最小元}。此元至多唯一。  
\end{definition}
\begin{definition}[最大元 (Greatest element)]
  若 $g\in P$ 满足对一切 $x\in P$ 均有 $x\le g$,则称 $g$ 为偏序集的\emph{最大元}。此元至多唯一。  
\end{definition}

\begin{definition}[极小元 (Minimal element)]
   若 $m\in P$ 且不存在 $x\in P$ 满足 $x<m$,则称 $m$ 为\emph{极小元}。偏序集中可有多个极小元。  
\end{definition}

\begin{definition}[极大元 (Maximal element)]
  若 $M\in P$ 且不存在 $x\in P$ 满足 $M<x$,则称 $M$ 为\emph{极大元}。偏序集中可有多个极大元。  
\end{definition}
      
\begin{definition}[上界 (Upper bound)]
   设 $S\subseteq P$。若 $u\in P$ 满足对所有 $s\in S$ 有 $s\le u$,则称 $u$ 为 $S$ 的\emph{上界}。  
\end{definition}

\begin{definition}[下界 (Lower bound)]
  设 $S\subseteq P$。若 $\ell\in P$ 满足对所有 $s\in S$ 有 $\ell\le s$,则称 $\ell$ 为 $S$ 的\emph{下界}。  
\end{definition}

\begin{definition}[最小上界 / 上确界 (Least upper bound / Supremum)]
  设 $S\subseteq P$,若 $s^* \in P$ 同时满足:
  \begin{enumerate}
    \item 对所有 $s\in S$ 有 $s\le s^*$;  
    \item 任意上界 $u$ 均满足 $s^*\le u$,
  \end{enumerate}
  则称 $s^*$ 为 $S$ 的\emph{最小上界}(上确界,supremum)。  
  \end{definition}
  
  \begin{definition}[最大下界 / 下确界 (Greatest lower bound / Infimum)]
  设 $S\subseteq P$,若 $\ell^* \in P$ 同时满足:
  \begin{enumerate}
    \item 对所有 $s\in S$ 有 $\ell^*\le s$;  
    \item 任意下界 $\ell$ 均满足 $\ell\le \ell^*$,
  \end{enumerate}
  则称 $\ell^*$ 为 $S$ 的\emph{最大下界}(下确界,infimum)。  
  \end{definition}


\begin{definition}
  设 $(P,\le)$ 为偏序集。如果对任意 $x,y\in P$,集合 $\{x,y\}$ 都有最小上界(并) $x\vee y$ 与最大下界(交) $x\wedge y$,
  则称 $(P,\le)$ 为\emph{格}。  
\end{definition}

\begin{proposition}[代数刻画]
  格可等价地定义为代数结构 $(L,\vee,\wedge)$,满足对所有 $a,b,c\in L$:  
  \[
  \begin{aligned}
  &\text{交换律: } a\vee b = b\vee a,\quad a\wedge b = b\wedge a;\\
  &\text{结合律: } (a\vee b)\vee c = a\vee(b\vee c),\quad (a\wedge b)\wedge c = a\wedge(b\wedge c);\\
  &\text{吸收律: } a\vee(a\wedge b)=a,\quad a\wedge(a\vee b)=a;\\
  &\text{幂等律: } a\vee a = a,\quad a\wedge a = a.
  \end{aligned}
  \]  
  此代数定义与偏序定义等价。  
\end{proposition}

\subsection{有界格与分配格}

\begin{itemize}
  \item 若格中存在最小元 $0$ 和最大元 $1$,称为\emph{有界格}。  
  \item 若对任意 $a,b,c\in L$,均满足分配律
  \[
    a\vee(b\wedge c)=(a\vee b)\wedge(a\vee c),\quad
    a\wedge(b\vee c)=(a\wedge b)\vee(a\wedge c),
  \]
  则称为\emph{分配格}。  
\end{itemize}