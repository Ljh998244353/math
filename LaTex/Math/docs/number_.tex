\section{整除}

\begin{proposition}
  令$a,b,c$为整数,那么有:
  $$
  \gcd(a+cb,b)=\gcd(a,b)
  $$
\end{proposition}

\begin{definition}
  $a,b,m,n\in \mathcal{Z}$,称$ma+nb$为$a,b$的线性组合
\end{definition}


\begin{theorem}
  裴蜀定理:\\
  如果$a,b$均为整数,则有整数$m$和$n$,使得
  $$
  ma+nb=\gcd(a,b)
  $$
\end{theorem}
其中该等式又被称为裴蜀等式,$m,n$被称为裴蜀数.\\
可以用扩展欧几里得算法求出$ma+nb=\gcd(a,b)$的特解,然后有通解
$$
\begin{cases}
  m&=m_0+k\frac{b}{\gcd(a,b)}\\
  n&=n_0-k\frac{a}{\gcd(a,b)}
\end{cases}
$$
注意到
$$
a(m+bu)+b(n-au)=\gcd(a,b)
$$
故满足等式的$m,n$有无穷多对.
\begin{lemma}
  两个不全为0的整数$a,b$的最大公因数是其线性组合中最小的正整数.
\end{lemma}

\begin{proof}
  不妨设$d$是$a,b$线性组合中最小的正整数.
  考虑带余除法:\\
  $a=dq+r$从而有
  $r=a-dq=a-q(ma+nb)=(1-qm)a-qnb$\\
  因此$d|a$,同理$d|b$,故$d$为公因数.\\
  不妨设$e=\gcd(a,b)$,那么$d|e$,又$e|(ma+nb)$,即$e|d$\\
  故$e=d$
\end{proof}

\begin{theorem}
  如果$a,b$是整数,那么所有 $a,b$ 的线性组合所构成的集合与所有
  $\gcd(a,b)$  的倍数所构成的集合相同.
  换言之,所有 $a,b$ 的线性组合,都是 $\gcd(a,b)$ 的倍数.
\end{theorem}

\begin{theorem}
  如果 $a,b$ 是不全为0的整数,那么正整数 $d$ 是 $a,b$的最大公因数,当且仅当
  1.$d|a,d|b$
  2.若$c|a,c|b$,那么$c|d$
\end{theorem}

\begin{definition}
  令 $a_1,a_2,\cdots ,a_n$为不全为0的整数,如果 $d$
 为他们公因子中最大的一个,则称 $d$
 为 $a_1,a_2,\cdots ,a_n$ 的最大公因数.
记为
  $ \gcd(a_1,a_2,\cdots ,a_n)=d$

\end{definition}

\begin{theorem}
$$
\gcd(a_1,a_2,\cdots ,a_n)=\gcd(a_1,a_2,\cdots, \gcd(a_{k-1},a_k),\cdots a_n)
$$
\end{theorem}


\begin{definition}
  我们称 $a_1,a_2,\cdots ,a_n $ 互素 如果 $ \gcd(a_1,a_2,\cdots ,a_n)=1$
\end{definition}

\begin{definition}
  我们称 $a_1,a_2,\cdots ,a_n $两两互素,如果任意两个数互素
\end{definition}

\begin{theorem}
  若$\gcd(a,m)=1$,$\gcd(b,m)=1$,则$\gcd(ab,m)=1$\\
  若$\gcd(a,b)=1$,则$\gcd(a^k,b^l)=1$
\end{theorem}

\begin{theorem}
  设正整数$a,b$之积 是一个整数的 $k(k\ge 2)$ 次幂.若
  $\gcd(a,b)=1$. 则 $a,b$都是整数的  $k$次幂.
  一般地:设正整数 $ a_1,a_2,\cdots ,a_n $ 之积是一个正整数的 $k$ 次幂.若 
  $a_1,a_2,\cdots ,a_n$ 两两互素,则 $a_1,a_2,\cdots ,a_n$
 都是整数的 $k$
 次幂.
\end{theorem}

\begin{lemma}
$$
\gcd(a_1^k,a_2^k,\cdots ,a_n^k)=gcd^k(a_1,a_2,\cdots ,a_n)
$$
\end{lemma}


\begin{corollary}
  裴蜀定理可以推广到 $n$ 个整数的情形:设 $a_1, a_2, \dots, a_n$
   是不全为零的整数,则存在整数 $x_1, x_2, \dots, x_n$, 使得 $a_1 x_1 + a_2 x_2 + \cdots + a_n x_n=\gcd(a_1, a_2, \dots, a_n)$。
   其逆定理也成立:设 $a_1, a_2, \dots, a_n$ 是不全为零的整数,$d > 0$ 是 $a_1, a_2, \dots, a_n$ 的公因数,
   若存在整数 $x_1, x_2, \dots, x_n$, 使得 $a_1 x_1 + a_2 x_2 + \cdots + a_n x_n=d$,则 $d = \gcd(a_1, a_2, \dots, a_n)$。
\end{corollary}

\begin{corollary}
  对自然数 $a、b$ 和整数$ n,a$ 与 $b $互素,考察不定方程:$ax+by=n$
 其中 $x$ 和 $y$ 为自然数。如果方程有解,称 $n$ 可以被 $a、b$ 表示。
 记 $C=ab-a-b$。由 $a$ 与 $b$ 互素,$C$ 必然为奇数。则有结论:
对任意的整数 $n$,$n$ 与 $C-n$ 中有且仅有一个可以被表示。
即:可表示的数与不可表示的数在区间 $[0,C]$ 对称(关于 $C$ 的一半对称)。$0$ 可被表示,$C$ 不可被表示;负数不可被表示,大于 $C$ 的数可被表示。
\end{corollary}

\begin{corollary}
  二元一次不定方程有非负整数解的条件\\
  $a,b>0$,若$ax+by=n$,$(a,b)=1$,则$n>ab-a-b$时有解,解的个数为$\left\lfloor \frac{n}{ab} \right\rfloor$
  $\left\lfloor \frac{n}{ab} \right\rfloor+1$ 
\end{corollary}

\newp{3}
一些小结论\\
  1.在[1e18]的范围下,一个数最多与连续7个数不互质.\\
  2.一个数能被4整除,当且仅当末尾两位能被4整除\\
  3.一个数能被25整除,当且仅当末尾两位能被25整除\\
  4. 一个数能被8整除,当且仅当末尾三位能被8整除\\
  5.一个数能被125整除,当且仅当末尾三位能被125整除\\
  6. 一个数能被3整除,当且仅当各位数之和能被3整除\\
  7. 一个数能被9整除,当且仅当各位数之和能被9整除\\
  8. 能被7整除的数的特征:a.抹去个位数b.减去原个位数的2倍c.其差能被7整除。\\
  9. 能被11整除的数的特征: a.抹去个位数 b.减去原个位数 c.其差能被11整除。或:奇数位上的数字和与偶数位上的数和相减,其差能被11整除
\section{同余}
以下所有参数未特殊说明,均为默认整数,模数默认正整数\\

\subsection{同余}

\begin{definition}
  设$m$是正整数,若$m|(a-b)$,则称$a$和$b$模$m$同余.记作$a \equiv b(\mod m)$
\end{definition}

性质\\
1. $a \equiv b (\mod m) \iff \exists k (k\in Z),a=b+kz$\\
2. $a \equiv a(\mod m)$\\
3. $a \equiv b(\mod m) \rightarrow b \equiv a(\mod m)$\\
4. $a \equiv b(\mod m),b \equiv c(\mod m) \rightarrow a \equiv c(\mod m)$\\
5. $a \equiv b(\mod m),c \equiv d(\mod m)\rightarrow a+c \equiv b+d(\mod m) ,a-c \equiv b-d(\mod m),ac \equiv bd(\mod m)$\\
6. $a^n \equiv b^n(\mod m)$\\
7. $ac \equiv bc (\mod m),d=\gcd(c,m)\rightarrow a \equiv b (\mod m/d)$\\
8. $a \equiv b (\mod m),n|m \rightarrow a \equiv b (\mod n)$\\
9. $a \equiv b (\mod m),a \equiv b (\mod n)\rightarrow a \equiv b (\mod lcm(m,n))$\\


\begin{definition}
  设模为$n$,则根据余数可将所有的整数分为$n$类,
  把所有与整数$a$模$n$同余的整数构成的集合叫做模$n$的一个剩余类,
  记作$[a]$。并把$a$叫作剩余类$[a]$的一个代表元。
\end{definition}

\begin{definition}
  从模$n$的每个剩余类中各取一个数,得到一个由$n$个数组成的集合,叫做模$n$的一个完全剩余系。
\end{definition}

\begin{proposition}
  若$r_1,r_2,\dots ,r_m$ 是模$m$的一个完全剩余系,且正整数$a$满足$\gcd(a,m)=1$,则对任何整数$b$,$ar_i+b$也为一个完全剩余类.
\end{proposition}

\begin{proof}
  若不然,则存在$ar_i+b \equiv ar_j+b (\mod m) \iff ar_i \equiv ar_j (\mod m) \iff m|a(r_i-r_j) \iff m|a_i-a_j \iff r_i \equiv r_j (\mod m)$与条件矛盾.故得证
\end{proof}


\begin{proposition}
  $a_1,a_2,\dots,a_m$ 是模$m$的一个完全剩余系,$b_1,b_2,\dots b_n$ 是模$n$ 的一个完全剩余系, 
  且$\gcd(n,m)=1$那么$na_i+mb_j$ 是模$mn$ 的一个完全剩余系
\end{proposition}

\begin{proof}
  首先由乘法原理知道有$mn$个数,那么只需证两两不同余即可\\
若不然则对于$(a,b)\neq(a',b')$
$na+mb \equiv na'+mb' (\mod mn) \iff mn|n(a-a')+m(b-b') \iff
m|(a-a'),n|(b-b') \iff a \equiv a' (\mod m),b \equiv b' (\mod n)$ 与条件矛盾.故得证.
\end{proof}

\begin{definition}
  简化剩余系也称既约剩余系或缩系,是$m$的完全剩余系中与$m$互素的数构成的子集,
  如果模$m$的一个剩余类里所有数都与$m$互素,就把它叫做与模$m$互素的剩余类。在与模$m$互素的全体剩余类中,
  从每一个类中各任取一个数作为代表组成的集合,叫做模$m$的一个简化剩余系。
\end{definition}

\begin{proposition}
  若$r_1,r_2,\dots ,r_m$ 是模$m$的一个缩系,且正整数$a$满足$\gcd(a,m)=1$,则$ar_i$也为一个缩系.
\end{proposition}


\begin{proof}
  由完系性质$1$可知其两两不同余,故只需证明其与均$m$互质即可.
因为$gcd(r_i,m)=1,gcd(a,m)=1\Rightarrow gcd(ar_i,m)=1$得证
\end{proof}

\begin{proposition}
  $a_1,a_2,\dots,a_m$ 是模$m$的一个缩系,$b_1,b_2,\dots b_n$ 是模$n$ 的一个缩系, 且$gcd(n,m)=1$那么$na_i+mb_j$ 是模$mn$ 的一个缩系
\end{proposition}

\begin{proof}
  只需证明其是所有与$mn$互质的剩余类.
$$
\begin{aligned}
  &\gcd(a_i,m)=1,\gcd(b_j,n)=1,\gcd(n,m)=1 \\&\Rightarrow \gcd(na_i,m)=1,\gcd(mb_j,n)=1\\  &\Rightarrow \gcd(na_i+mb_j,n)=1,\gcd(na_i+mb_j,m)=1\\& \Rightarrow \gcd(na_i+mb_j,mn)=1
\end{aligned}
$$
若将$a_i,b_j$扩展成完系,若$\gcd(na_i+mb_j,mn)=1 \Rightarrow \gcd(na_i+mb_j,m)=1,\gcd(na_i+mb_j,n)=1\Rightarrow \dots$ 逆着证回去即可.
\end{proof}


\subsection{线性同余方程}
\begin{definition}
  形如$ax \equiv b (\mod m)$ 的同余式称为一元线性同余方程
\end{definition}

\begin{theorem}
  $\gcd(a,m)=d,d\nmid b$,则无解,否则恰好有$d$个模$m$不同余的解.
\end{theorem}


\begin{proof}
  若$d\nmid b$,则 $ax \equiv b (\mod m)\iff ax-ym=b$,根据贝祖定理显然无解.\\
若$d\mid b$,则显然有无穷多组解,我们设其中一组特解为$x_0,y_0$ \\
其通解为$x=x_0+(m/d)t,y=y_0+(a/d)t$\\
设$x_1=x_0+(m/d)t_1,x_2=x_0+(m/d)t_2$\\
$x_1 \equiv x_2 (\mod m) \iff t_1 \equiv t_2 (\mod d)$\\
所以有$d$ 个不同于的解.
\end{proof}


\begin{definition}
  $\gcd(a,m)=1,ax \equiv 1 (\mod m)$ 则称该同余方程的一个解为$a$模$m$的逆,记为$a^{-1}$.
显然$\gcd(a^{-1},m)=1$
\end{definition}

\begin{theorem}
  设$p$ 为素数,正整数$a=a^{-1}$ ,当且仅当$a \equiv \pm 1(\mod p)$.
\end{theorem}


\begin{proof}
$a \equiv \pm 1(\mod p) \iff a^{2} \equiv 1(\mod p)$
反过来.有\\
$a^{2} \equiv 1(\mod p) \Rightarrow p|(a^2-1) \Rightarrow p|(a+1) 或p|(a-1) \Rightarrow a \equiv \pm 1(\mod p)$
\end{proof}

\begin{theorem}
  威尔逊定理\\
  若$p$是素数,则$(p-1)! \equiv -1 (\mod p)$
\end{theorem}

\begin{proof}
  $p=2$ 显然成立.\\
否则对于$1 \leq a \leq p-1$,可以找到其逆元与之配对,且除$1$和$p-1$都能两两配对.\\
故$(p-1)! \equiv p-1 \equiv -1 (\mod p)$
\end{proof}

\begin{theorem}
  威尔逊定理逆定理\\
  若$n\geq 2$是正整数,且$(n-1)! \equiv -1 (\mod n)$则 $n$为质数
\end{theorem}

\begin{proof}
  若不然,设$n$ 为合数,则其必存在小于$n$ 的素因子$p$\\
所以有$(n-1)! \equiv -1 (\mod n),p|n \Rightarrow (n-1)! \equiv -1 (\mod p)$,\\
但是 $(n-1)! \equiv 0 (\mod p)$\\
而$n>1$ 矛盾.故得证.
\end{proof}

\begin{theorem}
  费马小定理\\
  如果$p$是一个素数,$a$是正整数且$a$不是$p$的倍数,则$a^{p-1} \equiv 1 \pmod{p}$。
\end{theorem}

\begin{proof}
  因为$\gcd(a,p)=1$\\
所以 $\prod_{i=1}^{p-1} ia \equiv \prod_{i=1}^{p-1} i \pmod{p} \Rightarrow a^{p-1} \equiv 1 \pmod{p}$ 得证.
\end{proof}

\begin{definition}
  欧拉函数\\
  对于正整数 $n$,小于等于 $n$ 且与 $n$ 互质的正整数的个数,称为欧拉函数,记作$\phi{(n)}$
\end{definition}

\begin{theorem}
  设$m$ 是一个正整数,$a$ 是一个正整数且$\gcd(a,m)=1,那么a^{\phi(m)} \equiv 1 \pmod{m}$
\end{theorem}

\begin{proof}
  设$r_1,r_2,\dots r_{\phi(m)}$,是不超过$m$ 的模$m$的一个缩系.\\
那么$ar_1,ar_2,\dots ar_{\phi(m)}$也是一个缩系\\
故$ar_1ar_2\dots ar_n \equiv r_1r_2\dots r_n \pmod{m} \iff a^{\phi(m)} \equiv 1 \pmod{m}$\\
得证.
\end{proof}

\begin{definition}
  同余方程组是指一组形如下面的方程的集合:
$$
\begin{cases}
a_1 &\equiv b_1 \pmod{m_1} \\
a_2 &\equiv b_2 \pmod{m_2} \\
\vdots \\
a_n &\equiv b_n \pmod{m_n}
\end{cases}
$$

其中,$a_i$和$b_i$是整数,$m_i$是正整数。
这组方程要求对于每个$i$,$a_i$除以$m_i$的余数等于$b_i$除以$m_i$的余数,即$a_i$与$b_i$在模$m_i$下同余。
解同余方程组就是要找到满足所有这些条件的整数解。
\end{definition}

\begin{example}
  $$
\begin{cases}
x \equiv 1 \pmod{3} \\
x \equiv 2 \pmod{5} \\
\vdots \\
x \equiv 3 \pmod{7}
\end{cases}
$$

我们可以使用迭代法(逐级满足法)解决.
由第一个式子得 $x=3t+1$
然后带入$3t+1 \equiv 2 \pmod{5} \iff t \equiv 4\pmod{5}$
以此类推.
但是这只能解决一些简单的问题,下面我们给出一般地解法.
\end{example}


\begin{theorem}
  中国剩余定理(CRT)\\
  设$m_1,m_2,\dots ,m_r$ 是两两互素的正整数,则同余方程
$$
\begin{cases}
x \equiv a_1 \pmod{m_1} \\
x \equiv a_2 \pmod{m_2} \\
\vdots \\
x \equiv a_r \pmod{m_r}
\end{cases}
$$
有模$m_1m_2\dots m_r$ 的唯一解
$$
x=\sum_{i=1}^{r} a_iM_iM_i^{-1} 
$$
其中
$$
M_i= \frac{1}{m_i}\prod_{j=1}^r m_j ,\qquad M_iM_i^{-1} \equiv 1 \pmod{m_i}
$$
\end{theorem}

\begin{proof}
  先证明$x$是方程组的解.\\
对于任意一个方程有,\\
$x \equiv a_kM_kM_k^{-1}  \equiv a_k \pmod{m_k}$ ,显然成立.\\
下证唯一性.\\
若$x_1,x_2$ 为方程组的2个解,则有$x_1 \equiv x_2 \pmod{m_k} \iff m_k|(x_1-x_2) \iff M|(x_1-x_2) \iff x1 \equiv x2 \pmod{M}$
\end{proof}


\begin{theorem}
  拉格朗日定理\\
  $p$为素数,$f(x)=a_nx^n+a_{n-1}x^{n-1}+\dots +a_1x+a_0(p \nmid a_n)$ 
  是模$p$意义下的整系数多项式方程,则同余方程$f(x) \equiv 0 \pmod{p}$ 在模$p$ 意义下至多有$n$ 个不同的解.
\end{theorem}


\begin{corollary}
  若超过$n$个解,则$p \mid a_i(i=0,1,\dots n)$,即$f(x) \text{是模p意义下的零多项式}$
\end{corollary}

\begin{corollary}
  若$n \leq p$ 则同余式$f(x)=x^n+a_{n-1}x^{n-1}+\dots +a_1x+a_0 \equiv 0 \pmod{p}$有$n$ 
  个解的充要条件是$x^p-x$ 除以$f(x)$所得的余式的一切系数都是$p$ 的倍数
\end{corollary}
这里介绍一个比较重要的多项式,常用于构造\\
$$
f(x)=\prod_{i=1}^{p-1}(x-i) -(x^{p-1}-1)
$$ 

\begin{theorem}
  wolstenholme定理\\
  若$p$为大于3的素数,则
  $$
  \sum_{k=1}^{p-1} \frac{(p-1)!}{k} \equiv 0 \pmod{p^2} 
  $$
\end{theorem}



\section{乘性函数}

\begin{definition}
  算术函数\\
  定义在所有正整数上的函数称为算数函数.
\end{definition}

\begin{definition}
  乘性函数\\
  若$\gcd(m,n)=1$,均有 $f(mn)=f(m)f(n)$,则称$f$ 为乘性函数.
\end{definition}

\begin{proposition}
  若$f$为乘性函数,$n=\prod_{i=1}^{k} p_i^{a_i}$ 为一个素因数分解.则$f(n)=\prod_{i=1}^k f(p_i^{a_i})$
由定义显然成立.
\end{proposition}

\begin{definition}
  和函数\\
  $f$ 为一个算术函数,$F(n)=\sum_{d \mid n} f(d)$ 称为$f$ 的和函数
\end{definition}

\begin{definition}
  欧拉函数\\
  $\phi(n)=\sum_{i=1}^n [gcd(i,n)=1]$ ,称为欧拉函数.
\end{definition}

\begin{proposition}
  设$p$是素数,$\phi(p^a)=p^a-p^{a-1} $
\end{proposition}

\begin{proof}
  由定义,$\phi(p^a)= \sum_{i=1}^{p^a} [gcd(i,p^a)=1]= \sum_{i=1}^{p^a} 1-[gcd(i,p^a) \neq 1]=p^a- \sum_{i=1}^{p^a} [gcd(i,p^a) \neq 1]$,
  这样的$i$ 显然只有$p$的倍数,有$p^{a-1}$ 个,证毕.
\end{proof}

\begin{proposition}
  欧拉函数是乘性函数
\end{proposition}

\begin{proof}
  若$gcd(m,n)=1$
由缩系的定义知道,显然模$m$的缩系有$\phi(m)$个数,模$n$的缩系有$\phi(n)$个数,由缩系的一个性质知模$mn$ 的缩系有$\phi(m)\phi(n)$个数.故$\phi(mn)=\phi(m)\phi(n)$
\end{proof}

\begin{proposition}
  $n>2$,$\phi(n)$ 为偶数
\end{proposition}

\begin{proposition}
 $$
 \phi(n)=n\prod_{i=1}^k(1-\frac{1}{p_i})
 $$
\end{proposition}

\begin{proof}
  容斥原理可证明
这里用乘性函数的性质证明
$$
\begin{aligned}
  \phi(n)&=\prod_{i=1}^k \phi(p_i^{a_i})\\
  &=\prod_{i=1}^k(p^{a_i}-p^{a_i-1})\\
  &=\prod_{i=1}^{k} p_i^{a_i}\prod_{i=1}^k(1-\frac{1}{p_i})\\
  &= n\prod_{i=1}^k(1-\frac{1}{p_i})
\end{aligned}
$$
\end{proof}


\begin{proposition}
  欧拉函数的和函数
  $$
  F(n)=\sum_{d \mid n} \phi(d)=n
  $$
\end{proposition}

\begin{proposition}
  定义$C_d$ 为$1$到$n$中与$n$最大公因数为$d$ 的集合容易证明其是$1$到$n$构成的正整数集合的一个划分.
而$C_d$中有$\phi(n/d) $ 个元素,(若$a\in C_d$ 则$ gcd(a/d,n/d)=1 $),故$n=\sum_{d|n}C_d=\sum_{d|n}\phi(n/d)=\sum_{d|n}\phi(d)$ 得证.
\end{proposition}

\begin{definition}
  狄利克雷卷积\\
  $f$,$g$ 为算数函数,定义狄利克雷积为
  $$
  (f*g)(n)=\sum_{d|n}f(d)g\left(\frac{n}{d} \right)
  $$
\end{definition}
实际上这整数上一章的卷积在整除关系上的定义\\


性质\\
1. $f*g=g*f$\\
2. $(f*g)*h=f*(g*h) $\\
3. $f*(g+h)=f*g+f*h $\\

这里的算数函数就是$f(1,x)\in \mathcal{F}$\\
故有逆元的条件是$f(1)\neq 0$\\

\begin{theorem}
  如果$f$,$g$ 是乘性函数,则$f*g$也是乘性函数
\end{theorem}


\begin{theorem}
  若$F=f*g$,$h$ 是$g$ 的逆函数,那么$f=F*h$
\end{theorem}

\begin{theorem}
  乘性函数的和函数也是乘性函数
\end{theorem}

\begin{definition}
  因子和与因子个数函数\\
  $\sigma(n)=\sum_{d|n} d $\\
  $\tau(n)=\sum_{d|n}1 $
\end{definition}

\begin{proposition}
  因子和与因子个数函数均为乘性函数
\end{proposition}

\begin{proposition}
  设$n=\prod_{i=1}^{k} p_i^{a_i}$\\
  \[
    \begin{aligned}
      \sigma(n)&=\prod_{j=1}^{k} \frac{p_j^{a_j+1}-1}{p_j-1}\\
      \tau(n)&=\prod_{j=1}^{k} (a_j+1)
    \end{aligned}
  \]
\end{proposition}


\begin{definition}
  莫比乌斯函数\\
  \[
  \mu(n)=\begin{cases}
    1,& (n=1)\\
    (-1)^r,& (n=\prod_{i=1}^{r}p_i)\\
    0,& (other)
    \end{cases}
  \]
\end{definition}

\begin{theorem}
  $f=F*\mu$
\end{theorem}