\section{期望经典问题入门}
\url{https://notes.sshwy.name/Math/Expectation/Classic/#E-%E7%BB%8F%E5%85%B8%E9%A2%98}
重要公式与结论
1. 期望具有线性性\\
2. 独立事件的期望有
$$
E(XY)=EXEY
$$
3.
$$
E(X)=E(E(X|Y))
$$
\subsection{普通}
\begin{proposition}
  有 $n$ 个随机变量$\langle X_n \rangle$,每个随机变量量都是从[1,m]中随机一个整数,$\max{\langle X_n \rangle}$的期望
  为
  $$
  m-\frac{1}{m^n}\sum_{i=1}^{m-1}i^{n}
  $$。
\end{proposition}

\begin{proof}
  设$Y=\max{\langle X_n \rangle}$ , 有
  $$
  \begin{aligned}
    EY&=\sum_{1\le i \le m} P(Y=i)i\\
    &=\sum_{1\le i\le m} i(F_y(i)-F_y(i-1))\\
    &=\frac{1}{m^n}\left( \sum_{1\le i\le m} i^{n+1}-i\cdot (i-1)^{n}\right) \\
    &=\frac{1}{m^n}\left( \sum_{1\le i\le m} i^{n+1}-\sum_{0\le i\le m-1} i^{n+1} +i^{n}\right)\\
    &=\frac{1}{m^n}\left( m^{n+1}-\sum_{0\le i\le m-1}i^{n}\right)\\
    &=m-\frac{1}{m^n}\sum_{i=1}^{m-1}i^{n}
  \end{aligned}
  $$
\end{proof}

\begin{proposition}
  概率为$p$的事件期望$\frac{1}{p}$次发生.
\end{proposition}
\begin{proof}
  设随机变量$X$表示其在第$x$次发生. 
  $$
  \begin{aligned}
    EX&=\sum_{i=1}^{\infty} P(X=i)i\\
      &=\sum_{i=1}^{\infty} P(X=i) \sum_{j=1}^{i} 1\\
      &=\sum_{j=1}^{\infty} P(X\ge j)\\
      &=\sum_{j=1}^{\infty} (1-p)^{j-1}\\
      &=\sum_{j=0}^{\infty} (1-p)^j\\
      &=\frac{1}{1-(1-p)}\\
      &=\frac{1}{p}
  \end{aligned}
  $$
\end{proof}

\begin{proposition}
  现在红包发了一个$w$元的红包,有 $n$ 个人来抢(均匀分布)。那么请问第 $k$ 个人期望抢到$\frac{w}{2^k}$.
\end{proposition}

\begin{proof}
设第$k$个人抢到$X$,前面的人抢$Y$.
有
$$
E(X)=E(E(X|Y))=E(\frac{w-Y}{2})=\frac{w}{2}-\frac{1}{2}E(Y)
$$
然后容易解出答案.\\
离散情况这样难以求解,在取球一节有其他解法\\
\end{proof}

\begin{proposition}赠券收集问题\\
  一个$n$面的骰子,期望$nH_n$ 次能使得每一面都被掷到。
\end{proposition}

\begin{proof}
$t_x$为设已经出现了$x-1$面,掷出第$x$面的次数.\\
设$T=\sum t$.有
$$
E(T)=\sum_{i=1}^{n}E(t_i)
$$
由于掷出第$x$面的概率为$\frac{n-x+1}{n}$,固
$$
\begin{aligned}
  E(T)&=\sum_{i=1}^{n}E(t_i)\\
  &=\sum_{x=1}^{n} \frac{n-x+1}{n}\\
  &=n\cdot H_n
\end{aligned}
$$
\end{proof}
同时他们也是相互独立的.\\


\subsection{拿球}
\begin{proposition}
  箱子里有$n$个球$1,2,⋯,n$, 你要从里面拿$m$次球,拿了后不放回,取出的数字之和的期望为$\frac{m(n+1)}{2}$。
\end{proposition}

\begin{proof}
  设随机变量$x_i$:\\
  $$x_i=\begin{cases}
  i\quad&, \text{if i is chosen}\\
  0\quad&, \text{if i isn't chosen}
  \end{cases}
  $$
  那么有
  $$
  \begin{aligned}
    E\left(\sum_{i=1}^{n}x_i\right)&=\sum_{i=1}^{n}E(x_i)\\
    &=\sum_{i=1}^{n} \frac{m}{n} i\\
    &=\frac{m(n+1)}{2}
  \end{aligned}
  $$
  发现是否放回不影响期望\\
\end{proof}


\begin{proposition}
  箱子里有$n$个球$1,2,⋯,n$,你要从里面拿$m$次球,拿了后以$p_1$的概率放回,$p_2$
  的概率放回两个和这个相同的球(相当于增加一个球),取出的数字之和的期望为 $\frac{m(n+1)}{2}$。
\end{proposition}

\begin{proof}
  设$x_i$为第$i$个球的贡献, $y_i$ 为其被拿出来的次数 ,那么$x_i=i\cdot y_i$
  $$
  E\left(\sum_{i=1}^{n}x_i\right)=\sum_{i=1}^{n}E(y_i)\cdot i
  $$
  因为$E(y_i)=E(y_j)$, $\sum y_i =m$ 得出$E(y_i)=\frac{m}{n}$
  故上式答案为$\frac{m(n+1)}{2}$
\end{proof}

\subsection{游走}
\begin{proposition}
  在一条$n$个点的链上游走,从一端走到另一端的期望步数为$(n-1)^2$。
\end{proposition}

\begin{proof}
  假设步数是$S$,求$S$的期望。我们定义一个随机变量$x_i$表示从i出发随机游走,第一次到$i+1$的步数。
  $$
  E\left(\sum_{i=1}^{n-1}x_i\right)= \sum_{i=1}^{n-1}E(x_i)
  $$
  又
  $$
  EX_{i+1}=\frac{1}{2}+\frac{1}{2}(1+EX_i+EX_{i+1})
  $$
  故
  $$
  EX_{i+1}=EX_{i}+2
  $$
\end{proof}

\begin{proposition}
  在一个$n$个点的完全图上游走,从一个点走到另一个点的期望步数为$n-1$。
\end{proposition}

\begin{proof}
  由于是完全图, 所以任意两个点为起点和终点的期望步数是相同的.于是有
  $$
  E=\frac{1}{n-1}+\frac{n-2}{n-1}(1+E)
  $$
  得到$E=n-1$
\end{proof}

\begin{proposition}
  在一个$n$个点的完全二分图$K_{n,n}$上游走,从一个点走到另一个点的期望步数为$2n-1$。
\end{proposition}

\begin{proof}
  $E_1$表示异侧点的期望,$E_2$表示同侧点的期望。
  $$
  \begin{aligned}
    E_1&=\frac{1}{n}+\frac{n-1}{n}(1+E_2)\\
    E_2&=1+E_1
  \end{aligned}
  $$
  故 $E_1=2n-1 ,E_2=2n$
\end{proof}

\begin{proposition}
  在一个$n$个点的菊花图上游走,从一个点走到另一个点的期望步数为$2n-3$。
\end{proposition}

证明与上类似\\



\subsection{解题方法}
1.贡献法,\\
若不行可以尝试更换计算贡献的东西(如边->点)\\